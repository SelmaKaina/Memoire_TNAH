Au cours de ce mémoire, nous avons exposé les principaux enjeux liés à l'automatisation du traitement des reportages photographiques de la Présidence de la République aux Archives nationales. Ces enjeux incluent l'analyse du contenu et du contexte de production, l'identification de l'indexation comme préoccupation centrale, la description du cadre normatif et institutionnel du chantier de reprise des données, ainsi que la présentation de l'outil de création de paquets d'archives. Pour conclure, nous proposons de prendre du recul afin de décontextualiser ces étapes et d’en extraire un processus plus général, potentiellement applicable à d’autres fonds et services d’archives.

Nous avons d'abord identifié les différentes sources d'informations susceptibles d'enrichir le signalement des documents traités. Cela a nécessité la mise en place de méthodes pour consulter et exporter les informations contenues dans les fichiers ainsi que celles provenant d'autres documents, notamment les outils de description archivistique existants. Cette quête de sources d'informations issues de contextes variés nous a amenés à approfondir notre réflexion sur l'importance de l'indexation dans la gestion documentaire, en particulier dans un contexte archivistique. Nous avons démontré en quoi les caractéristiques d’un signalement varient en fonction des besoins des professionnels qui les produisent et du contexte de production. Pour évaluer l’impact de l’indexation sur l’accessibilité du fonds, nous avons étudié les besoins des utilisateurs devant naviguer dans ce fonds. En ajoutant une indexation thématique, détachée des logiques traditionnelles de production et de classement archivistique, notre objectif est de permettre aux utilisateurs de dépasser le regroupement conventionnel des archives, basé sur la logique du service producteur, en offrant la possibilité de recherches thématiques croisées entre différents dossiers d’un même fonds.

Ensuite, la définition du contexte institutionnel, des normes de l’archivage électronique et des standards d’échanges de données nous a permis d’identifier précisément la forme finale attendue de nos traitements : la structure des paquets d’archives numériques requise par le système d’archivage électronique. Nous avons souligné l’importance de diviser les fonds en ensembles homogènes pouvant être traités de manière uniforme et de définir une méthode spécifique pour chaque ensemble, en identifiant également les processus communs à traiter.

Dans la dernière partie de ce mémoire, nous avons décrit le processus de création d'un outil permettant d'automatiser les traitements nécessaires pour parvenir au résultat souhaité. Cela impliquait de modéliser les processus et de les adapter au langage de programmation utilisé, tout en tenant compte des contraintes et des opportunités offertes. La production d’un tel outil, dans un contexte très particulier, a souvent nécessité de développer des solutions sur mesure, parfois improvisées, pour résoudre les problèmes rencontrés en cours de route.

L'automatisation du traitement archivistique d'un fonds ne peut se réduire à la simple conception d’un outil automatisant un ensemble de processus. Cette étape finale repose sur une analyse approfondie des aspects techniques, historiques, et institutionnels du fonds, ainsi que sur une compréhension des besoins du service chargé de sa gestion.
\\

Quel avenir peut-on envisager pour un outil répondant à un besoin aussi spécifique que la reprise des reportages photographiques de la Présidence de la République ? Comme nous l'avons montré dans la dernière partie de ce mémoire, l’application ORPhÉE n’est pas adaptée à l’empaquetage de tous les fonds de photographies numériques dans le contexte Vitam : elle ne permet même pas de reprendre l’ensemble des reportages de la Présidence, en raison des différences de méthodes de classement entre les reportages traités par la méthode Constance et ceux non traités. Le guide d’utilisateur\footnote{Voir le \href{https://github.com/SelmaKaina/ORPhEE/blob/main/README.md}{guide de l'application} sur Github.} décrit les prérequis nécessaires au bon fonctionnement de l’application, destiné à guider les archivistes du Département de l’administration des données, et pourrait être utilisé par des services ayant des fonds très similaires à traiter. Toutefois, en concevant l’outil sous forme de pipeline de données, composé d’une série de processus distincts et indépendants, on peut envisager la réutilisation de certains modules dans d’autres contextes et pour d’autres fonds. En adaptant ces modules aux besoins spécifiques, ils pourraient être intégrés dans une architecture plus appropriée au fonds concerné. À terme, la diversité des fonds traités permettrait de constituer un catalogue de processus distincts, dont la réutilisation permettrait de s’adapter à une gamme toujours plus large de problématiques archivistiques. Dans le contexte actuel de production massive de données sous des formats variés, les possibilités d'accélération des traitements offertes par ces méthodes d'automatisation représentent un gain considérable pour les services d'archives.