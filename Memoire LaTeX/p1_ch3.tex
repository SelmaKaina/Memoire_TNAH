\chapter{Les enjeux du traitement de la photographie comme document d'archive : description, évaluation, communicabilité}
\label{sec:chap3}
L'analyse d'un fonds archivistique nécessite de s'intéresser à son contenu intellectuel, en grande partie déterminé par les activités du service producteur, à sa matérialité et à ses caractéristiques techniques, comme nous venons de le faire, mais également à la tradition archivistique associée à cette typologie documentaire. Cette tradition n'existe pas encore pour les photographies nativement numériques, aussi devons-nous nous référer dans un premier temps aux pratiques relatives au traitement archivistique des photographies argentiques. Il conviendra, au fil de cette réflexion, d'évaluer la pertinence de ces pratiques pour l'archivage des photographies nativement numériques. Nous pouvons nous inspirer des méthodologies établies par de grandes institutions, telles que les Archives nationales françaises\footcite{missionphotographiqueDescriptionDocumentsPhotographiques2019} ou le Bureau canadien des archivistes\footcite{bureaucanadiendesarchivistesReglesPourDescription1990}, pour guider notre réflexion. En tant qu'archives iconographiques et nativement numériques, les photographies nativement numériques se distinguent des archives textuelles, mais aussi des archives photographiques argentiques. Elles nécessitent un traitement archivistique spécifique en raison de la richesse des informations qu'elles contiennent. Cette complexité les rend souvent difficiles à trier, à décrire et à classifier de manière adéquate.

\section{La difficile définition de critères de tri}

Les étapes d’évaluation et de description sont profondément interconnectées dans le processus archivistique, chacune influençant directement l'autre. Avant l'archivage intermédiaire et historique, la sélection des documents à conserver repose sur une description préliminaire qui permet d'évaluer leur intérêt fonctionnel, juridique et historique, particulièrement pour les versements de grande envergure ne pouvant être parcourus dans leur intégralité dans un délai raisonnable. Après cette phase de sélection, les documents conservés pour leur valeur historique sont à nouveau décrits dans un instrument de recherche. Aux Archives nationales, toutes les archives versées sont destinées à une conservation définitive, éliminant ainsi la nécessité d'une réévaluation. En effet, l'évaluation est réalisée en amont, généralement par les missions des Archives au sein des ministères. Toutefois, il est essentiel d'examiner les critères qui régissent la sélection des archives photographiques afin de comprendre le fonds et d'identifier les enjeux archivistiques spécifiques qui orienteront son traitement.

\subsection*{Des critères pour déterminer la valeur documentaire et informationnelle d'une image}

Dans leur ouvrage consacré à la gestion des archives photographiques, Normand Charbonneau et Mario Robert, archivistes québecois, proposent un ensemble de critères de sélection à l’usage des archivistes confrontés à un fonds de photographies argentiques\footcite[p.55]{charbonneauGestionArchivesPhotographiques2001}. Ces critères d’évaluation sont répartis en plusieurs catégories : ceux liés à l’information contenue dans les documents, ceux liés à l’utilisation des photographies, et ceux relatifs à leurs propriétés physiques.

L’intérêt informationnel des photographies réside dans leur capacité à refléter les activités du service producteur et à apporter des informations permettant d’approfondir la compréhension de ces activités. Les informations fournies doivent être suffisamment rares et pertinentes pour illustrer le fonctionnement du service producteur de manière originale. La qualité esthétique est également considérée comme un critère lié à la valeur intellectuelle de la photographie, puisqu’une image de mauvaise qualité (flou, mauvais cadrage) ne permettra pas une interprétation pertinente du contenu informationnel du document. De plus, les aspects esthétiques peuvent refléter un courant ou une pratique qui inscrit le document dans l’histoire de l’art et des techniques photographiques\footcite[pp.102-103]{charbonneauGestionArchivesPhotographiques2001}. 

Les attentes des lecteurs sont également prises en compte dans la définition de ces critères de tri. En effet, dans le contexte d’un archivage intermédiaire, il est nécessaire de déterminer qui peut avoir besoin de consulter les archives et dans quel contexte. Une connaissance approfondie des usages associés à ces documents permettra d’anticiper au mieux les futures demandes et d'étendre notre compréhension de l’intérêt du fonds au-delà de sa valeur historique. 

Enfin, les propriétés physiques du document renvoient à sa valeur en tant que support d’information : s’agit-il d’un support rare ou particulièrement ancien ? Le support est-il de bonne qualité et susceptible d’être conservé de manière pérenne ? La qualité du support entrave-t-elle l’intelligibilité de l’information ? Si l’information est inutilisable ou que son coût de récupération est disproportionné par rapport à la valeur informationnelle du document, il peut être préférable de ne pas le conserver\footcite[pp.53-55]{charbonneauGestionArchivesPhotographiques2001}.

\subsection*{Difficultés propres au tri des photographies}

La gestion des fonds photographiques, en tant qu'archives iconographiques, présente des défis particuliers, exacerbés lorsqu'une collection est principalement composée de ce type de documents. Le tri devient particulièrement complexe en raison du manque d'expérience des archivistes avec ce format spécifique, ainsi que de la difficulté à évaluer la valeur informationnelle des images. Cette problématique est aggravée par l'avènement de la photographie numérique, qui a considérablement augmenté le volume de clichés produits. Contrairement à la photographie analogique, où les images étaient limitées par les contraintes matérielles et nécessitaient un choix préalable pour le développement, la numérisation donne l'illusion d'une gestion simplifiée : les images n'occupent plus d'espace physique et sont immédiatement accessibles.

L’organisation des versements photographiques et les outils de gestion d’archives tendent à favoriser une évaluation à un niveau de description élevé, comme le dossier, mais laissent les archivistes relativement démunis lorsqu'il s'agit de trier les photographies individuellement\footcite[p.49]{charbonneauGestionArchivesPhotographiques2001}. Cette approche reflète celle adoptée dans le contexte des reportages photographiques de la Présidence de la République, où la valeur archivistique des documents est déterminée au niveau du reportage entier. En pratique, seuls les reportages jugés non essentiels sont éliminés avant leur versement aux Archives nationales, sans qu'une évaluation fine des clichés individuels ne soit réalisée.

Au-delà des contraintes techniques et méthodologiques, le rapport mémoriel que nous entretenons avec les photographies complique encore davantage ce travail de tri. De plus, il semble difficile, voire impossible, d'épuiser le contenu informationnel d'une image, et aux yeux d'un photographe maîtrisant les outils de retouche d'image, tout cliché a une valeur potentielle\footcite[p.12]{bergheindexation}. Ainsi, la conservation de toutes ces images semble justifiée par leur potentiel documentaire, nourrissant une logique de rétention qui alourdit considérablement les fonds photographiques. D’un point de vue archivistique, ce potentiel documentaire infini présente un dilemme. Si la suppression de clichés jugés moins pertinents peut sembler nécessaire pour gérer la masse, elle risque aussi de compromettre l'objectif archivistique de refléter fidèlement la production du service photographique. En effet, bien que les utilisateurs se concentrent souvent sur le contenu visuel immédiat, les archivistes doivent considérer la photographie dans son ensemble, comme une pièce d’un puzzle plus large, représentant non seulement un instant capturé, mais aussi un témoignage du fonctionnement du service producteur\footcite[p.101]{charbonneauGestionArchivesPhotographiques2001}.

\section{Méthodes et enjeux de la description et de l'indexation des archives photographiques}

Contrairement à l’adage selon lequel une image vaut mille mots, une photographie ne peut être exploitée sans une description textuelle qui en précise le contenu. Il est indispensable de retranscrire par écrit le contenu des images pour les rendre interrogeables et compréhensibles par l’archiviste comme par le lecteur. Les règles de description des documents d’archives sont conçues pour répondre à des questions essentielles (quoi, par qui, quand) et pour normaliser ces descriptions afin de faciliter leur exploitation.\footcite[pp.102-103]{charbonneauGestionArchivesPhotographiques2001}. 

Le titre d’une photographie, ou d’un groupe de photographies, peut être attribué à différents moments de la vie du document : par son auteur lors de sa création, par des détenteurs intermédiaires, par le service producteur, ou encore par le service d’archives lors du classement. Dans le cas des photographies argentiques, le titre peut être inscrit directement sur le document ou figurer sur un support annexe, comme une enveloppe. En archivage électronique, le titre devient une métadonnée, soit interne, intégrée au fichier, soit externe, présente dans un autre document. Pour les reportages photographiques de la Présidence de la République, par exemple, la description archivistique se fait au niveau du reportage entier, avec un titre qui est une métadonnée externe consignée dans des documents annexes comme l’instrument de recherche ou l’agenda officiel du Président. Certaines photographies incluent également des éléments descriptifs dans leurs métadonnées internes, renseignés par la cellule photographique.

L’indexation permet d’associer les documents ou dossiers à des termes, liés à leur contenu ou à leur producteur afin d’en faciliter l’accès. Lorsque l’indexation découle de la description, elle est tributaire de son degré de profondeur et de son exhaustivité. Pour garantir une description cohérente et uniforme, le service d’archives doit définir à l’avance le niveau de précision adopté et choisir entre un vocabulaire libre ou contrôlé.  Une description parfaitement exhaustive d’une photographie étant quasiment impossible, le service d’archives doit décider des éléments à indexer, tels que les personnalités, les lieux ou les monuments\footcite[p.155]{charbonneauGestionArchivesPhotographiques2001}. Le recours à un vocabulaire contrôlé, bien que restrictif, assure la cohérence des termes employés, tandis qu’un vocabulaire libre permet plus de flexibilité, mais peut entraîner des incohérences si la méthode n’est pas rigoureuse. L’utilisation de fiches d’autorité est recommandée pour standardiser les noms propres, éviter les confusions et assurer l’interopérabilité des métadonnées\footcite[pp.158-170]{charbonneauGestionArchivesPhotographiques2001}.

Cependant, les logiques d'accès aux fichiers ne suivent pas forcément les logique de classement. Par exemple, dans le cadre de l'archivage des reportages photographiques de la Présidence de la République, les photographies sont regroupées par événement, logique qui reflète la production des fichiers par le service. Les usagers des Archives nationales souhaitant consulter ce fonds demandent rarement les photographies d'un événement, mais sont plus restrictifs : ils cherchent les photographies du Président avec telle personne, avec tel objet (voitures présidentielles), ou à tel endroit. Afin de répondre à ce type de demande, en l'absence d'indexation fine, les archivistes doivent tenter de faire des liens entre les personnes, les objets et les lieux, susceptibles d'apparaître au cours de tel reportage.
\\

Dans la perspective de faciliter l'indexation des archives photographiques nativement numériques, l'utilisation des métadonnées internes descriptives peut sembler une solution séduisante. Pour ce faire, il ne s'agit pas de modifier ces métadonnées internes : non seulement cela irait à l'encontre du principe de préservation de l'intégrité du fonds, mais il ne suffit pas de les modifier pour les rendre interrogeables. En effet, afin d'être requêtées, elles doivent être extraites et stockées dans une base de données visant à interroger les fichiers. L’indexation par le service d’archives doit donc être réalisée directement dans un logiciel de gestion ou via des documents annexes. Ces questions, ainsi que les avantages et inconvénients de l'indexation par le service producteur, seront abordés dans le prochain chapitre où nous présenterons les pratiques descriptives du service photographique de la Présidence de la République et du service des archives.

\section{Déterminer la communicabilité des reportages photographiques de la Présidence de la République}

Une image peut être interprétée de mille manières différentes, en particulier lorsqu’elle est sortie de son contexte de production. Ce risque de mauvaise interprétation, voire de détournement malveillant, est amplifié aujourd'hui par les avancées en intelligence artificielle, qui permettent de générer et retoucher des images avec une facilité déconcertante. Lorsqu'il s'agit de personnalités publiques, notamment politiques, ce risque est encore accru, car il peut conduire à la propagation de campagnes de désinformation.

Les photographies numériques sont souvent nommées de manière incrémentale, sans utiliser un vocabulaire interrogeable, ce qui complique leur identification et leur gestion, notamment lorsqu'il s'agit de filtrer ou de restreindre l'accès à certaines images. Un processus efficace de signalement des images sensibles nécessiterait une identification précise des clichés au contenu potentiellement problématique afin d'y associer des règles de communicabilité restrictives. Or, dans le cadre des reportages photographiques de la Présidence de la République, la masse des fichiers rend une telle entreprise virtuellement impossible.

\subsection*{Les reportages de la Présidence de la République : des archives librement communicables ?}

Les reportages de la cellule photographique de la Présidence de la République avaient initialement été classés comme librement communicables en vertu des articles L.213-1 à L.213-6 du code du patrimoine. Ces archives, produites par un service public et sans signalement particulier du service producteur, étaient supposées être communicables de plein droit\footcite{articleL2131CodePatrimoine}. Cependant, une analyse plus approfondie révèle que cette classification a été établie sans une évaluation détaillée de son contenu.

Les photographes suivent le Président de la République lors de la majorité de ses déplacements, à l'Élysée comme lors d'événements officiels, et documentent parfois des moments privés de son agenda. Il en résulte des clichés représentant des personnes, des lieux ou des situations susceptibles d'être soumis à des dérogations aux dispositions de l'article L.213-1, mentionnées dans l'article L.213-2. Les reportages dits \enquote{privés} sont ainsi soumis à un délai de communicabilité de cinquante ans au titre de la protection de la vie privée. Ce même délai s'applique également aux reportages dont le contenu pourrait porter atteinte à la sûreté de l'État, tels que ceux documentant les réunions de crise suite aux attentats du 13 novembre 2015\footcite{articleL2132CodePatrimoine}. De plus, les déplacements du Président, notamment ses visites dans des établissements scolaires, peuvent inclure des photographies en présence de mineurs, sans garantie d'une autorisation  écrite par un responsable légal. Conformément au respect du droit à l'image des mineurs et à la protection de la vie privée, le délai de communicabilité de ces reportages doit également être prolongé à cinquante ans.


\subsection*{L'application du droit d'auteur aux archives photographiques}

En vertu de l’Article L. 112-1 du Code de la propriété intellectuelle, le droit d’auteur définit les droits dont un auteur dispose sur ses \oe{}uvres et lui permet de décider de la manière dont elles seront utilisées. Le droit d’auteur s’applique aux \oe{}uvres de l’esprit : elles doivent avoir pris forme sur un support et être originales, en reflétant les choix créatifs de leur auteur. Bien que les limites de cette définition puissent parfois sembler un peu floues dans le contexte de la photographie documentaire, le choix du cadrage et du moment de la prise de vue dans les reportages photographiques constituent des choix créatifs propres au photographe, faisant de lui l'auteur du cliché.  Ainsi, les photographies réalisées par les agents de la cellule photographique de la Présidence de la République peuvent bien être considérées comme des \oe{}uvres originales de l'esprit. 

Le droit d’auteur se divise en deux catégories : le droit moral, qui confère à l’auteur le respect de son nom et de son \oe{}uvre, et les droits patrimoniaux qui permettent de contrôler l’exploitation des \oe{}uvres et d’obtenir une contrepartie financière. Les droits patrimoniaux s’étendent, à quelques exceptions près, jusqu’à soixante-dix ans après la mort de l’auteur. Tandis que les droits patrimoniaux peuvent être vendus ou cédés, le droit moral est perpétuel et imprescriptible.

Pour déterminer les potentielles restrictions dues au droit d'auteur pour les reportages photographiques de la Présidence de la République, nous devons nous interroger sur les applications de ce droit aux \oe{}uvres produites par les agents de service public dans l'exercice de leurs fonctions. En effet, jusqu'à la mandature d'Emmanuel Macron, les photographes employés par la cellule photographique étaient exclusivement des agents de service public, titulaires ou contractuels. Les droits patrimoniaux relatifs aux \oe{}uvres produites par des agents du service public sont détenus non pas par leur auteur mais par le service en ayant commandé la création. En revanche, le droit moral est inaliénable et s'applique bien aux \oe{}uvres originales des agents de service public. L'exercice de ce droit dans son intégralité est pourtant susceptible de paralyser l'exploitation des productions du service et donc d'entraver ses missions. Ainsi, \enquote{pour les \oe{}uvres créées dans l’exercice des fonctions ou d’après les instructions reçues, les prérogatives morales, à l’exception du droit de paternité, sont paralysées}\footcite{touboulDroitsAuteurAgents2011}. Au regard de ces considérations, le nom du photographe doit autant que possible être associé aux clichés qu'il a créés, sans pour autant que son absence puisse limiter la communication ou la diffusion des reportages photographiques. 
\\

À la lumière des éléments évoqués dans cette section -- à savoir le respect du droit à l'image,  de la vie privée,  et du secret défense -- les reportages photographiques de la Présidence de la République sont considérés comme non librement communicables. En raison du manque de temps et de ressources humaines pour attribuer un délai de communicabilité spécifique à chaque reportage, le délai de cinquante ans est appliqué uniformément par mesure de précaution. Par conséquent, l'accès à ces reportages n'est possible que sur demande de dérogation, ou après qu'une demande ait déclenché une réévaluation de la communicabilité d'un reportage. Dans ce cas, une analyse approfondie du reportage concerné est effectuée pour en déterminer la communicabilité. De plus, ces reportages ne sont ni diffusables ni réutilisables, sauf après un examen minutieux par les agents des Archives nationales, dans l'éventualité où un usager solliciterait l'autorisation de diffusion des images consultées.
\\

Comme nous l’avons montré dans les chapitres précédents, deux des principaux enjeux de l’archivage des photographies numériques sont la gestion de la masse et l’accessibilité des informations représentées. En effet, l’absence de contenu textuel directement accessible, les nommages souvent non signifiants et la volumétrie des reportages photographiques en font des documents particulièrement difficiles à analyser et rend la recherche d’informations presque impossible. De plus, dans le contexte de l’archivage historique aux Archives nationales, il ne peut être question d’une nouvelle évaluation qui, au-delà des difficultés propres au tri des photographies, ne relève tout simplement pas des prérogatives de l’institution. 

L’indexation est donc au c\oe{}ur de la gestion de ces archives. Au regard de la quantité de fichiers, une absence de description interrogeable est une limite considérable qui rend la recherche thématique de fichiers virtuellement impossible. L’indexation est donc la solution la plus indiquée pour appréhender la masse des photographies numériques et contourner les contraintes volumétriques. Cependant, il n’existe pas d’indexation idéale et universelle : le choix des informations à indexer et le vocabulaire choisi  dépendent entièrement de l’institution à l’origine de l’indexation et répondront à ses propres besoins métiers ; besoins qui peuvent varier grandement d’une institution à l’autre, et qui peuvent évoluer au cours de la vie du document numérique. 
