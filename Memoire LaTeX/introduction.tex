Dès 1965, dans son essai consacré aux usages sociaux de la photographie, Pierre Bourdieu écrit :
\begin{quotation}
	\enquote{On s’accorde communément pour voir dans la photographie le modèle de la véracité et de l’objectivité : \enquote{Toute \oe{}uvre d’art reflète la personnalité de son auteur, lit-on dans l’Encyclopédie française. La plaque photographique, elle, n’interprète pas. Elle enregistre. Son exactitude, sa fidélité ne peuvent être remises en cause” […] En réalité, la photographie fixe un aspect du réel qui n’est jamais que le résultat d’une sélection arbitraire, et, par là, d’une transcription..}\footcite[p.108]{bourdieuArtMoyenEssai1965}.}
\end{quotation}

Bourdieu exprime ici l'idée que la photographie est largement perçue comme un médium réaliste car produit mécaniquement et offrant donc une représentation objective du réel, indépendante des choix de son auteur, et donc supposément non sujette à interprétation, contrairement aux autres arts graphiques où la \enquote{patte} de l’auteur est beaucoup plus visible. 

La réalité est encore plus complexe, car au-delà de la subjectivité du photographe mise en avant par Bourdieu à la fin de cette citation, il faut prendre en compte aussi celle de l’observateur. Sur la subjectivité inhérente à l'interprétation des photographies, Gisèle Freund écrit dix ans plus tard :

\begin{quotation}
	\enquote{La photographie du chef d'État, porté dans des cortèges et des démonstrations, surplombant des assemblées, ou ornant les bureaux officiels, est pour les uns le symbole du père, pour les autres celui du Grand Frère orwellien. Elle inspire l'amour ou la haine, la confiance ou la peur. Sa valeur intrinsèque réside dans sa puissance d'éveiller des émotions.\footcite[p.205]{freundPhotographieSociete2006}}
\end{quotation}

En effet, une photographie, bien que capturant un instant précis, peut être perçue de multiples façons selon le contexte dans lequel elle est présentée et le point de vue de l'observateur. Les choix faits par le photographe -- comme l'angle de prise de vue, le cadrage, la composition, et même l'instant choisi pour déclencher l'appareil -- influencent profondément la façon dont l'image sera interprétée.

Ces deux citations illustrent le contraste entre la perception populaire de la photographie et sa réalité en tant que médium. Ainsi, le lien de corrélation entre ces idées réside dans cette tension : alors que la photographie est souvent perçue comme une vérité figée, elle possède en réalité une grande malléabilité interprétative. Cette dualité entre la croyance en sa nature objective et son potentiel de manipulation ou de subjectivité rend la photographie à la fois puissante et complexe, car elle peut être utilisée pour véhiculer des messages très divers, parfois même contradictoires, tout en conservant une aura d'authenticité.

Ces spécificités rendent les photographies particulièrement attrayantes pour les médias et les services de communication, tout en constituant une source d'informations extrêmement riche pour les chercheurs. Il est donc essentiel pour les archivistes non seulement de les rendre accessibles, mais aussi de veiller à ce que leur contexte de production soit documenté et restitué avec rigueur. Cette préoccupation n'est certes pas récente, mais se présente aujourd'hui avec une intensité nouvelle en raison de la production exponentielle de photographies nativement numériques et de leur inclusion croissante dans les versements d'archives.

L'objectif de mes deux stages au Département de l'administration des données en 2023 et 2024, était de proposer des méthodes de reprise des reportages de photographies numériques de la Présidence de la République et des services du Premier Ministre en vue de leur versement dans la nouvelle plateforme d'archivage électronique des Archives nationales. En première année, mes activités se sont concentrées sur le développement d'une méthode semi-automatique de reprise des reportages des services du Premier ministre. Mon stage de fin d'étude s'inscrit dans la continuité du premier : j'étais chargée de développer une méthode automatique de fabrication de paquets d'archives pour les reportages de la Présidence de la République. Dans le cadre de ce mémoire, je m'attacherai à présenter les connaissances et compétences que j'ai dû mobiliser et développer afin d'effectuer cette mission. Nous nous intéresserons donc aux enjeux de l'automatisation du traitement archivistique des photographies nativement numériques à travers l'exemple de la reprise des reportages photographiques de la Présidence aux Archives nationales.

Dans un premier temps, nous nous intéresserons au contexte de production des photographies et à leurs caractéristiques matérielles. En effet, le fait qu'il s'agisse d'archives numériques ne nous exempt pas d'une connaissance approfondie des caractéristiques de cette typologie documentaires. Tout comme un archiviste traitant un fonds de parchemins enluminés se doit de connaître le processus de fabrication du support, les encres utilisées et le langage dans lequel l'auteur s'exprime, un archiviste chargé d'un fonds de photographies numériques doit pouvoir expliciter les caractéristiques techniques des fichiers qu'il traite : supports de stockage, structure du train binaire et des pixels, format des fichiers, métadonnées embarqué, et encodage des données textuelles. 

La dimension immatérielle des archives produites par les technologies numériques représente un défi inédit pour les archivistes. Elle impose de se familiariser avec des concepts et des modes de fonctionnement souvent éloignés de ceux propres à l'archivage papier. En outre, l'archivage numérique ne peut se faire de manière aussi \enquote{directe} que l'archivage papier : alors qu'un archiviste peut accéder physiquement et immédiatement à un fonds papier, l'archivage numérique implique l'intermédiation par une interface technologique. Cette interface, qui interprète une partie des données, crée une distance inévitable entre le professionnel et le fonds qu'il traite. Cette situation est d'autant plus complexe que l'archiviste, souvent formé dans les sciences humaines plutôt qu'en informatique, peut ne pas comprendre entièrement le fonctionnement de cette interface, qui est pourtant cruciale pour l'accès aux informations archivées. L'archiviste n'est toutefois pas démunis devant cette différence de forme, puisque les principes fondamentaux de l'archivage sont applicables au contexte numérique. Les enjeux de description, de classement, et de préservation de l'intégrité des fonds s'appliquent en effet à la gestion des fonds de données numériques.

Dans la deuxième partie de ce mémoire, nous explorerons le cadre normatif et institutionnel de l'archivage numérique qui régit les traitements appliqués aux fonds et les méthodes de versement. Nous commencerons par présenter les normes en vigueur qui encadrent le fonctionnement des systèmes d'archivage électronique ainsi que les méthodes pour la création de paquets d'archives numériques. Nous conclurons cette deuxième partie par une présentation du chantier de reprise des données aux Archives nationales, dans lequel s'inscrit le processus de reprise des reportages photographiques de la Présidence. La \enquote{reprise des données} est le processus par lequel des informations, issues de systèmes anciens, sont extraites, transformées, et réintégrées dans un nouveau système. Dans le contexte de l'archivage numérique aux Archives nationales, il s'agit du transfert des données archivées selon la méthode mise en place par le programme Constance vers la nouvelles solution d'archivage électronique développée par Vitam. 

Ces deux premières parties fourniront une vue d'ensemble des contraintes et des enjeux associés à la gestion de ce fonds dans le contexte actuel. Le développement d'une application pour automatiser la reprise des reportages photographiques m'a conduit à examiner les opportunités offertes par les technologies numériques ainsi que les défis techniques liés à l'automatisation des processus modélisés. L'application que j'ai développée fonctionne selon le principe d'un pipeline de données, c'est à dire comme une chaîne de traitement automatisée. Les données, provenant de diverses sources, entrent dans ce pipeline où elles subissent une série d'opérations successives. Chaque étape du processus est conçue pour transformer, nettoyer, ou enrichir les données, de manière à les rendre compatibles et prêtes pour leur destination finale : ici, le système d'archivage électronique des Archives nationales. Le fonctionnement de ce pipeline de données sera détaillé dans la troisième partie de ce mémoire.

Certains outils, comme l'application Resip\footcite{resip}, sont déjà conformes aux normes et permettent de créer des paquets d'archives acceptés par le système d'archivage électronique des Archives nationales. Cependant, la valeur ajoutée de mon travail réside dans le développement d'une application en \gls{python} capable non seulement de créer ces paquets, mais aussi d'y ajouter des métadonnées descriptives issues de sources variées. Ces métadonnées enrichissent le signalement des reportages et des fichiers au sein du système d'archivage, améliorant ainsi leur accessibilité. Ainsi, l'enjeu principal de cette automatisation est d'assurer l'accessibilité du fonds versé. Cet enjeu est particulièrement important dans le cadre du versement d'un fonds volumineux constitué exclusivement de photographies numériques : il ne s'agit en effet pas d'indexer des photographies au sein d'un versement d'archives bureautiques, mais d'indexer un fonds presque entièrement composé de photographies numériques. En effet, sans un  signalement efficace, un fonds aussi vaste que celui des reportages photographiques de la Présidence de la République risque de devenir inaccessible en pratique.