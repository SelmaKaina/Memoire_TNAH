\chapter{Les archives du Service photographique de la Présidence de la République, reflet d'une transition technologique}

Dans ce premier chapitre, nous allons brièvement retracer l'histoire du service photographique de la Présidence de la République. Cette histoire, qui remonte au tout début du XX\textsuperscript{ème} siècle n'a jamais été relatée dans son entièreté, les ouvrages et instruments de recherche se concentrant uniquement sur une présentation du service à un moment donné. Ainsi, la présentation qui suit est le fruit d'une enquête dont la piste suivait l'histoire des versements de reportages photographiques de la Présidence aux Archives nationales. Il n'a pas toujours été possible d'établir une cohérence entre les différentes sources d'informations, certaines se contredisaient même parfois. C'est en mettant en regard un grand nombre d'instruments de recherches, de notices producteur et de sources bibliographiques que j'ai pu établir une chronologie qui a ensuite été confirmée par les archivistes du services des archives de la Présidence.

\section{Documenter l'activité des chefs d'État français : la création d'un Service photographique de la Présidence de la République }

Le suivi photographique des chefs d'État français est une préoccupation bien antérieure à la création du service photographique de la Présidence en 1952. Dès 1929, leurs activités étaient déjà documentées par le service central photographique du ministère de l'Intérieur. Cette continuité, s'étendant donc de 1929 à nos jours, permet une étude diachronique des pratiques de documentation photographique des chefs d'État.

\subsection*{Photographier tant les criminels que les présidents : le service central photographique et d'identité du ministère de l'Intérieur}

Le service central photographique du ministère de l'Intérieur est rattaché au service central d'Identification au sein de la Direction générale de la Sûreté nationale. Ce service voit le jour en 1907, plus de vingt ans après le service central d'Identité judiciaire fondé en 1893\footcite{chaveReportagesPhotographiquesPresidence2016, batReportagesPhotographiquesAutour2016}. L'objectif des deux services est identique : systématiser l'identification criminelle grâce à la méthode de l'anthropométrie judiciaire développée par Alphonse Bertillon à la préfecture de Police de Paris. Initialement basée sur des portraits parlés, l'identification évolue rapidement vers la photographie face-profil. Le développement de cette méthode dans les années 1880 s'inscrit dans un contexte plus large de recours à la photographie comme instrument scientifique d'enregistrement du visible, rendu possible par les perfectionnements techniques des décennies 1870-1880, en particulier le développement des plaques au gélatino-bromure d'argent\footcite[p.138]{bajacPhotographieDaguerreotypeAu2010}. La création du service central photographique du ministère de l'Intérieur s'inscrit dans le cadre d'un ensemble de réformes initiées par Georges Clemenceau dans le but de renforcer et moderniser les services de la police. Chargé dès sa création de l'exécution des travaux photographiques du ministère, il se voit en outre confier la couverture des déplacements du chef de l'État en province, mission qu'il assure jusqu'à la création d'un service dédié à l'Élysée\footcite{batReportagesPhotographiquesAutour2016}.

\subsection*{La création du Service photographique de la Présidence}

En 1952, à la demande du président Vincent Auriol, un service photographique est créé à l'Élysée. Les photographes sont alors des agents détachés de la préfecture de Police auprès des services de sûreté de la Présidence et doivent réaliser la couverture photographique des événements et sorties du Président, officiels comme privés, ayant lieu à l'Élysée et en région parisienne\footcite{batReportagesPhotographiquesAutour2016}. Ils peuvent également être sollicités pour des photographies d'urgence en fonction de l'emploi du temps du chef de l'État, que ce soit pour immortaliser les réceptions des personnalités reçues ou les divers événements ayant lieu à l'Élysée. Les reportages peuvent être commandés par le service de presse, le service du protocole ou le chef du cabinet. À l'exception des reportages privés et de certaines commandes spécifiques des services de l'Élysée, les photographes, souvent des gardes républicains, couvrent les mêmes événements que les photographes d'agences accrédités\footcite{perez-bastiePresidenceJacquesChirac2015}. C'est seulement à partir de mai 1971 que le Service photographique de la Présidence prend le relais du service central de la photographie du ministère de l'Intérieur et se voit également confier la couverture des déplacements du Président en province et à l'étranger\footcite{bouillonRepresentationsCharlesGaulle2016}. Les liens entre les deux services demeurent très forts, les agents du Service photographique de la Présidence ayant longtemps été exclusivement des gendarmes ou gardes républicains.


\section{L'avènement de la photographie numérique et les évolutions du Service photographique de la Présidence}

Au sein du Service photographique de la Présidence, la transition vers les supports numériques s'effectue progressivement entre 2003 et 2005, durant la seconde mandature de Jacques Chirac. En 2004, seuls 19 reportages sont réalisés en argentique, et en 2005, un seul. Par ailleurs, entre 2003 et 2005, 19 reportages bénéficient d'une double couverture numérique et argentique. Le Service photographique du Premier Ministre connaît une évolution similaire : lors d'un échange en été 2023, Benoît Granier, photographe à Matignon depuis vingt ans, témoignait du caractère hautement expérimental et transitionnel de cette période. Il raconte comment cette double couverture pouvait être effectuée par un même photographe, alternant entre appareil argentique et numérique, conscient de la nécessité de se former à ce nouvel outil qui promettait de transformer la pratique photographique.

La transition qui s'effectue au Service photographique de la Présidence entre 2003 et 2005 est alors tout à fait représentative d'une évolution plus globale qui sonne le glas de l'appareil photographique argentique dans la pratique courante. Pour les photographes professionnels, comme ceux du Service photographique de la Présidence, ainsi que pour les amateurs, cette révolution numérique entraîne une transformation significative des pratiques photographiques. L'augmentation des capacités de stockage des appareils numériques et des espaces de sauvegarde, ainsi que la possibilité de supprimer les clichés ratés, conduit à une explosion du nombre de fichiers produits et conservés. La fonction de prise de vue en rafale, rendue possible par les appareils numériques, donne l'impression de garantir au moins une image réussie parmi de nombreuses prises. Cette perception incite souvent à privilégier la quantité sur la qualité, ce qui se traduit fréquemment par une accumulation de photographies de qualité médiocre. De plus, la rapidité avec laquelle les photos peuvent être publiées et communiquées a considérablement changé. Grâce aux téléphones portables et à la transmission via Internet, les images peuvent être partagées presque instantanément. La sélection des clichés les plus \enquote{remarquables} se fait désormais de manière beaucoup moins posée et réfléchie, souvent quelques minutes seulement après la prise de vue.

\subsection*{La cellule photographique du Service de l'audiovisuel et de l'organisation technique des déplacements}

En 2007, le Service photographique de la Présidence est rattaché au Service de l'audiovisuel et de l'organisation technique des déplacements (SAOTD) et renommé \enquote{cellule photographique}. La cellule photographique dispose d'un effectif variant de trois à quatre photographes, complétés par un ou deux iconographes chargés de gérer et d'indexer l'ensemble des prises de vue. Entre 2007 et 2012, 8 photographes se sont succédés au sein de la cellule photographique\footcite{archivesnationalesFrancePresidenceRepublique}. 

Il est cependant difficile d'expliciter clairement le fonctionnement de la cellule photographique au fil de la succession des mandatures. Au cours d'un échange avec Evelyne Van Den Neste, cheffe du Service des archives et de l'information documentaire de la Présidence de la République, et son adjoint Cyrille Chareau, nous avons pu lever le voile sur certains aspects fonctionnels non explicités dans les notices producteurs et instruments de recherche des Archives nationales. La Présidence de la République ne passant pas par des décrets pour fixer l’organisation de ses services, les évolutions nombreuses sont difficiles à documenter, et la lisibilité de l’administration dans son ensemble est complexe. En raison du renouvellement régulier des équipes, au rythme des changements de cabinets et de chefs d’État, la mémoire des services ne s'inscrit pas dans le long terme : peu de personnes peuvent témoigner des pratiques antérieures à leur arrivée, ce qui peut entraîner un manque de continuité dans les pratiques. Cette spécificité impacte particulièrement le service photographique : peu de personnes sont aujourd’hui capables de décrire le fonctionnement du service lors de la mandature précédente. La nomenclature \enquote{officielle} de ce service étant difficile à obtenir, nous nous y référerons sous les appellations de \enquote{service photographique} ou \enquote{cellule photographique} de la Présidence. 

La cellule photographique est aujourd’hui rattachée au service de la communication, et est donc très impactée par les enjeux politiques liés à l'élaboration d'une image présidentielle. Les photographes du service doivent couvrir l’intégralité de l’agenda du Président et s’adapter aux évolutions d’un emploi du temps susceptible de changer à la dernière minute. La cellule photographique se retrouve dans une position ambivalente : créée pour documenter l’activité du Président, ses clichés ne correspondent pas toujours aux critères esthétiques définis par le service de la communication. Cet objectif purement documentaire est tout à fait exceptionnel et peut être difficile à justifier auprès des autres services. Il s’inscrit dans une pratique bien antérieure à la communication sur les réseaux sociaux, très exigeante en matière de qualité esthétique. Les photographies produites par la cellule ne sont pas utilisées par la presse, qui envoie ses propres photographes lors des réceptions et déplacements présidentiels. De plus, bien que la cellule photographique soit rattachée au service de la communication, ses clichés ne semblent pas toujours être exploités par ce dernier. Par exemple, lors d'un chantier de reprise des données d'archives numériques de la Présidence de la République aux Archives nationales, une page internet du site de la présidence française de l'Union Européenne (ue2008.fr) a été identifiée, présentant une photographie de François Hollande qui ne faisait pas partie du reportage photographique de l'événement, malgré la présence de nombreux clichés de cet événement, pris sous un angle très similaire. Selon les archivistes du service des archives et de l'information documentaire de la Présidence de la République, les photographies sont largement utilisées à des fins documentaires, notamment pour comparer le protocole suivi par les prédécesseurs du Président lors de certains événements, ou pour la production de cadeaux diplomatiques sous la forme d'albums photographiques destinés à d'autres chefs d'État.

Le premier versement de reportages photographiques numériques de la Présidence de la République aux Archives nationales a eu lieu en mars 2007. Ce fonds faisait partie d'un versement plus large de l'ensemble des archives électroniques et audiovisuelles de la Présidence Jacques Chirac, produites entre 1995 et 2007\footcite{perez-bastiePresidenceJacquesChirac2015}. Avant leur versement, les reportages avaient été gravés sur supports CD ou DVD par la cellule photographique, tandis qu'une autre partie des photographies avait été versée sur le serveur des Archives nationales. Les reportages conservés sur CD et DVD ont été transférés sur serveur après leur extraction et traitement au Centre du Microfilm d'Espeyran. Les reportages des mandatures suivantes ont été versés sur disques-durs externes aux Archives nationales.

\section{Description du fonds : contenu, typologie et volumétrie}

\subsection*{La capture du quotidien des présidents de la République}

Les reportages photographiques suivent le quotidien des chefs de l'État. Ils documentent les événements publics, mais aussi certains événements privés de l'agenda présidentiel : réceptions, cérémonies, remises de décorations, visites diplomatiques, réunion du conseil des ministres, sorties officielles du Président, pots de départ... Les occasions ne manquent pas. Les reportages sont numérotés, cependant il est difficile d'en faire un décompte exacte : l'agenda chargé des photographes entraîne parfois une numérotation un peu chaotique. Certains numéros sont sautés et certains répétés plusieurs fois. On trouve par exemple après le reportage 152696, un reportage 152696bis, puis 152696bis1 et enfin 152696bis2. Toutefois, entre 2005 et 2017, environ 8000 reportages ont été produits, soit plus d'un par jour en moyenne. Bien sûr, le rythme de production n'est pas régulier : la mandature de Nicolas Sarkozy compte presque moitié moins de reportages que celle de François Hollande. De plus, deux reportages peuvent renvoyer à des durées complètement différentes, de quelques minutes à plusieurs jours. Pour cette raison, il est vain d'établir un nombre moyen de clichés par reportage : certains ne comptent qu'un fichier, d'autres des centaines. La volumétrie des fonds donne une meilleure idée de la quantité totale de fichiers : les reportages de la mandature de François Hollande représentent un volume de 2,6 To, ceux de la mandature de Nicolas Sarkozy un volume de 1,3 To, et ceux de la mandature de Jacques Chirac plus de 600 Go.

\subsection*{Une typologie liminale : les numérisations de photographies argentiques}

En 2019, lors d'un chantier de reconditionnement aux Archives nationales des photos argentiques du Service photographique de la Présidence  Jacques Chirac, un ensemble de neuf reportages a été trouvé sur des CD, dont trois n'étaient pas décrits dans l'instrument de recherche du versement des archives électroniques de cette mandature. Plusieurs des fichiers issus de ces reportages ont pu être identifiés comme étant des numérisations de photographies argentiques. Cette découverte illustre bien l'évolution des méthodes de travail d'un service au cours d'une période de transition où coexistent et se croisent deux pratiques distinctes.  À l'époque où ce CD a été gravé, la communication et/ou le traitement des photographies passaient par des outils numériques, nécessitant ainsi la numérisation des photographies argentiques.

Cette transition pose des questions importantes sur le traitement archivistique des numérisations. La principale interrogation est de savoir si les numérisations doivent être considérées comme des originaux. En effet, si une version physique de ces archives (comme un négatif ou un tirage) existe toujours, il n'est généralement pas nécessaire de conserver la version numérisée si elle n'a pas été modifiée et si les raisons de sa numérisation ne sont pas documentées. En revanche, si l'original physique n'est plus disponible, la numérisation doit être conservée, mais elle doit être clairement identifiée comme telle, par exemple en utilisant une indexation spécifique. Il est crucial de la différencier des photographies nativement numériques, car leur traitement archivistique diffère de manière significative :
\begin{enumerate}
    \item Du point de vue archivistique, comme évoqué précédemment, une numérisation de photographie argentique ne correspond pas à un original mais plutôt à une version de communication. 
    \item D'un point de vue technique, les informations de date associées à la création d'une numérisation et à celle d'un fichier nativement numérique renverront à des périodicités différentes. Un fichier numérique résultant d'une numérisation est créé lors du processus de numérisation et sa date de création reflète ce moment, et non pas la date originale de la prise de vue de la photographie argentique. En revanche, la date de création d'une photographie nativement numérique correspond à la date à laquelle le fichier a été créé dans l'appareil, ce qui coïncide avec la date de la prise de vue\footnote{La date de création enregistrée par l'appareil photographique ne doit pas être confondue avec la date de création du fichier stockée par le système d'exploitation Windows. Celle-ci peut ne pas correspondre à la date de prise de vue, mais plutôt à la date de copie ou d'enregistrement du fichier, selon la manière dont il a été enregistré dans l'ordinateur.}.
\end{enumerate}

Cette distinction est essentielle pour assurer une gestion et une conservation appropriées des archives photographiques, permettant ainsi de préserver non seulement la valeur historique des images, mais aussi leur intégrité documentaire.
\\

La présentation du service producteur et du contenu du fonds constitue une étape essentielle de tout travail archivistique. Dans ce contexte, les fonctions du service justifient le volume considérable de fichiers à traiter. Étant donné que l'objectif principal du service est de documenter l'activité des présidents, plutôt que de produire des clichés à des fins communicationnelles ou esthétiques, et qu'il est chargé de suivre le Président tout au long de son agenda, cela conduit inévitablement à une production massive de fichiers. Le passage à la photographie numérique n'a fait qu'accentuer ce phénomène, celle-ci permettant une production encore plus importante et un stockage facilité par son apparente immatérialité. De l'absence de sélection esthétique ou sémantique il résulte des versements de fonds qui ne semblent pas avoir été triés\footnote{On peut noter que certains reportages de la Présidence de la République ont un sous-dossier \enquote{internet} ou \enquote{sélection} qui témoigne de l'existence d'un processus de tri. Cependant, ces dossiers ne sont pas toujours présents et constituent une part très minime des versements :  la collecte des photographies de ce service ne peut donc pas se réduire à cette sélection.}.