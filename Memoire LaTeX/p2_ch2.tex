\chapter{La reprise des données pour un versement dans le nouveau système d'archivage électronique des Archives nationales}

Pour conclure le chapitre précédent, nous avons exploré les problématiques liées à l’indexation, en mettant en lumière les divergences entre les intérêts métier, les pratiques d’utilisation et les contraintes rencontrées dans le traitement des reportages photographiques de la Présidence de la République. Nous avons vu comment celles-ci se traduisent en défis concrets pour la manipulation des archives, notamment en raison des différences de pratiques et des contextes de production. Dans ce chapitre, nous allons approfondir ces questions en examinant comment elles s'articulent avec le projet de reprise des données en vue de leur versement dans le nouveau système d'archivage électronique (SAE) Vitam des Archives nationales. En effet, une connaissance de ce contexte normatif et institutionnel est indispensable pour comprendre ce qu'il permet en termes d'indexation et d'exploitation des métadonnées internes. De plus, cela nous permettra d'approfondir notre compréhension du fonds des reportages photographiques, dans la mesure où son traitement actuel ne peut être décorrélé des traitements antérieurs menés par les Archives nationales.

Pour cela, nous nous attacherons à présenter le cadre normatif et l'environnement dans lequel s’effectuent les versements des archives numériques, ce qui nous permettra de mieux comprendre les enjeux techniques propres aux Archives nationales. Cette analyse inclura une présentation des normes encadrant l'archivage électronique, ainsi qu'un examen du SAE des Archives nationales et de la manière dont le chantier de reprise des données s’inscrit dans ce contexte institutionnel et technique. Il s'agit ici de définir les contraintes spécifiques qui impactent la gestion et la préservation des archives numériques aux Archives nationales, et de comprendre comment les reportages photographiques de la Présidence s'intègrent dans ce cadre complexe.

\section{Le cadre de l’archivage électronique}

Dans cette section, nous allons examiner le cadre normatif et les principes fondamentaux de l'archivage électronique, ce qui nous permettra de mieux comprendre les contraintes et les spécificités associées à la gestion des archives numériques. Nous commencerons par définir ce qu'est un système d'archivage électronique (SAE), en nous appuyant sur les normes et les exigences qui le régissent. Ensuite, nous présenterons les modèles conceptuels clés, notamment l'Open Archival Information System (OAIS) et le Standard d'Échange de Données pour l'Archivage (SEDA). Sur ces bases, nous pourrons mieux saisir comment ces cadres influencent la gestion, la conservation et l'accès aux documents numériques aux Archives nationales.

\subsection*{Le système d'archivage électronique : une brique dans le système d'information archivistique}

Le Système d’Archivage Électronique (SAE) est une sous-composante spécialisée du Système d’Information (SI), conçue spécifiquement pour gérer les archives numériques. Un Système d’Information (SI) est un ensemble de ressources permettant de collecter, stocker, traiter et diffuser de l’information, structuré selon les besoins de l’institution qui l’utilise. Dans le cadre des services d’archives, ce SI est désigné par le terme Système d’Information Archivistique (SIA). Le SIA constitue l’infrastructure générale pour la gestion des archives, englobant les outils et procédures nécessaires à la gestion des documents, qu'ils soient physiques ou numériques. Il est souvent composé de plusieurs briques fonctionnelles, dont la brique de base est le logiciel-métier permettant la gestion des archives. Ce socle doit garantir la collecte, le classement, la conservation et la communication des documents\footcite{bothSystemeInformationArchivistique2020}.

Lorsque les services d’archives traitent des masses importantes de documents numériques, l'intégration d'un SAE au sein du SIA devient indispensable. Le SAE doit répondre aux exigences spécifiques définies par la norme NF Z 42-013, publiée par l'AFNOR en 1999 et révisée en 2000 puis en 2009. Cette norme établit les exigences et recommandations fonctionnelles, organisationnelles et d’infrastructure nécessaires pour la conception et l’exploitation d’un SAE. Contrairement aux systèmes de gestion électronique de documents (GED), les SAE sont adaptés aux exigences particulières de la gestion des documents d’archives. Ils doivent garantir la disponibilité, l’intégrité, la confidentialité des documents ainsi que la traçabilité des opérations. Conformément à cette norme, le SAE doit également gérer les éléments de preuve associés aux documents archivés, incluant la production et l’archivage de journaux quotidiens horodatés, qui intègrent des éléments de preuve tels que les empreintes uniques des fichiers, les dates et heures des opérations réalisées.

Pour assurer la gestion des archives électroniques, le SAE doit inclure des fonctionnalités de collecte, de conservation et de communication des documents numériques. Pour assurer une intégration efficace, des mécanismes doivent être établis pour permettre aux applications du SIA de déposer des documents électroniques dans le SAE ou de consulter ceux déjà versés. Cela implique la mise en place d'une interface et de protocoles de communication entre les deux systèmes\footcite{wakimSAESystemesStockage2011}. Plutôt que de conservation, dans le cas des documents numériques, on préfère le terme de pérennisation. Contrairement aux documents physiques qui nécessitent une conservation matérielle, les données numériques ne se détériorent en soi, mais peuvent subir des pertes brutales et irréversibles en cas de dégradation des supports. Des opérations de vérification des empreintes numériques permettent de garantir l'intégrité des fichiers et de détecter toute altération survenue lors des opérations de gestion au sein du SAE. Le SAE doit également gérer la conservation de copies, qui, dans le contexte numérique, ont valeur d'original\footnote{Voir chapitre \ref{sec:chap3} pour la définition de la notion d'original dans le contexte de la photographie nativement numérique}. La communication passe par la gestion des accès aux documents numériques. On distingue trois types de communication : publique (les lecteurs en salle), interne (le personnel des Archives) ou administrative à l’extérieur, lorsqu’un service demande communication d’un de ses dossiers déjà versé. Dans le contexte des Archives nationales, sur lequel nous reviendrons dans le prochain chapitre, le SAE permet la consultation par les archivistes et services versants par un système de cloisonnement, réservant l'accès aux archives aux utilisateurs autorisés. Les demandes de consultation du public sont gérées par une autre brique du SIA des Archives nationales.
\\

Les processus de versement, de conservation, de communication, de restitution et d’élimination des documents doivent être conformes aux principes définis par la norme OAIS (ISO 14721).

\subsection*{Présentation de la norme OAIS}

Le modèle de référence OAIS (Open Archival Information System), élaboré par le Consultative Committee for Space Data Systems en 2002 et standardisé par l'ISO en 2012, constitue un cadre conceptuel pour la gestion, l'archivage et la préservation à long terme des documents numériques. Il définit les concepts et éléments de base offrant une vue globale et cohérente de l'archivage numérique. En s'appuyant sur le formalisme UML (\gls{uml}), l'OAIS propose un modèle, indépendant de toute application particulière. L'architecture qu'il dessine est destinée à garantir l'accessibilité et l'intelligibilité des informations archivées au fil du temps.

Le modèle définit quatre principaux acteurs impliqués dans le processus d'archivage\footcite[pp.41-44]{banat-bergerArchivageNumeriqueLong2009}: 

\begin{enumerate}
\item L'archive, définie comme \enquote{une organisation chargée de conserver l'information pour permettre à une communauté d'utilisateurs cible d'y accéder et de l'utiliser}.
\item Le producteur, qui fournit l'information à conserver. Il n'est pas nécessairement le producteur de l'information et est plutôt associé à la notion de \emph{service versant} dans le contexte archivistique.
\item L'utilisateur, désigne ici une personne ou un système entrant en relation avec \emph{l'archive} pour rechercher et consulter l'information conservée.
\item Le management, qui représente les décideurs chargés de déterminer le mandat, les priorités et les orientations de l'archive, en cohérence avec la politique de l'institution. Souvent à l'origine des sources de financement, il peut donc décider de l'orientation des ressources et évaluer les performances de \emph{l'archive}.
  \end{enumerate}


Ces acteurs s'échangent de l'information sous la forme de paquets d’information, contenant les objets à archiver et les métadonnées nécessaires à leur visualisation et pérennisation. Nous avons présenté dans le chapitre 2 la forme des fichiers numériques, constitués d'une suite de 0 et de 1. Dans le modèle OAIS, ce fichier numérique est modélisé par l'objet \emph{contenu d'information}, constitué d'un \emph{objet données} qui ne peut être interprété qu'à l'aide \emph{d'informations de représentation}. Parmi ces informations de représentation, les \emph{informations de structure} permettent d'interpréter les séquences de bits et de les traduire en caractères ou, dans le cas des images numériques, en pixels ; tandis que les \emph{informations sémantiques} fournissent des éléments permettant de comprendre la signification des données, par exemple l'unité de mesure dans laquelle des données chiffrées seraient exprimées\footcite[p.45]{banat-bergerArchivageNumeriqueLong2009}. Ainsi, les informations de représentation doivent permettre de représenter et de comprendre le contenu d'information. Les informations de pérennisation quant à elles permettent à l'archive d'assurer ses responsabilités. Elles contiennent des informations de contexte, de provenance, d'identification (association d'un identifiant à chaque objet numérique) et d'intégrité (l'empreinte numérique du fichier)\footcite[pp.48-49]{banat-bergerArchivageNumeriqueLong2009}. 

Enfin, les informations de description correspondent aux métadonnées descriptives des archives, obtenues à partir d'une analyse des objet numériques et de leurs informations de représentation et de pérennisation. Elles ne font pas partie du contenu d'information, mais permettent aux utilisateurs de rechercher les données au sein du système d'archivage électronique\footcite[p.172]{rietschDematerialisationArchivageElectronique2006}. Les informations d’empaquetage  (Packaging  Information) permettent de mettre  en  relation  les  différents composants du paquet d'information, c'est-à-dire l'objet données et les informations associées.

L’OAIS spécifie trois types de paquets d'information : le SIP (Submission Information Package), fourni par le producteur et remis au service d'archives ; le AIP (Archival Information Package), qui est conservé au sein du SAE ; et le DIP (Dissemination Information Package), qui est mis à disposition des utilisateurs pour consultation. Le modèle décrit également les interactions entre ces acteurs, incluant le versement des objets numériques par les producteurs, leur stockage à long terme sous forme d’AIP par le service d'archives, et la fourniture des documents aux utilisateurs sous forme de DIP. Les systèmes d'archivage électronique (SAE), comme Vitam, illustrent l'application des principes du modèle OAIS. 

\subsection*{Le SEDA}
Le Standard d'échange de données pour l'archivage (SEDA) est le fruit d'une collaboration initiée en 2006 entre les Archives de France et l'ancienne direction générale de la modernisation de l'État (DGME), dans le cadre du programme ADELE (Action pour le Développement de l'Administration Électronique). L'objectif du SEDA est de faciliter l'interopérabilité entre le système d'information d'un service d'archives et les systèmes d'information de ses partenaires, comme les services producteurs et versants, lors de l'échange de données. Le SEDA repose sur des normes et standards préexistants, avec comme structure de base la norme ISO 14 721, aussi connue sous le nom de modèle OAIS (Open Archival Information System). Le langage \gls{xml} a été retenu pour structurer les informations dans ce standard. Le SEDA identifie cinq acteurs principaux susceptibles d'intervenir dans ces échanges : le service versant (TransferringAgency), le service producteur (OriginatingAgency), le service d'archives (ArchivalAgency), le service de contrôle (ControlAuthority), qui peut intervenir pour valider les transactions ; et enfin, le demandeur d'archives (Requester), qui peut être toute personne physique ou morale souhaitant consulter les archives conservées\footcite{sibilleStandardEchangeDonnees2015}.

Ce standard permet l'échange de paquets d'information, tels que définis dans le modèle OAIS, en distinguant l'archive, qui regroupe le contenu des données, les informations de représentation et les informations de pérennisation. Selon le SEDA, un paquet d'informations à verser (SIP) est constitué d'un bordereau de transfert et d'un ou plusieurs objets à archiver. Ce bordereau se trouve à la racine du SIP et décrit l'ensemble des métadonnées du paquet, comprenant un en-tête, une déclaration des objets binaires, une description des archives représentées par ces objets, des métadonnées descriptives et de gestion, et les identifiants du service versant et du service d'archives\footcite[p.14]{programmevitamStructurationSubmissionInformation2023}. La description du contenu (ContentDescription) permet de décrire l’Archive et ses subdivisions intellectuelles en lui associant des informations de description et d’indexation\footcite{sibilleStandardEchangeDonnees2015}.
\\

Bien que le SEDA détaille les processus de transfert de données numériques, il ne spécifie pas les règles de constitution des paquets à transférer. La structure du paquet et du bordereau dépend donc en grande partie du système d'archivage électronique utilisé. Lors de la conception du pipeline de données destiné à la reprise des reportages photographiques, nous nous sommes référés non seulement au dictionnaire du SEDA\footcite{siafseda}, mais nous avons également dû adapter l'outil aux exigences du SAE des Archives nationales. 


\section{L'archivage électronique aux Archives nationales : des années 1980 à nos jours}

Après avoir exploré le cadre normatif qui régit l'archivage numérique en France, il est essentiel de contextualiser ces principes au sein des Archives nationales. Cette section explore le contexte spécifique de l'institution, en retraçant l’évolution de l’archivage électronique depuis les premières initiatives avec le programme Constance jusqu’à l’adoption du système d’archivage électronique Vitam. À travers cette analyse, il s’agit de comprendre comment ces évolutions institutionnelles et technologiques ont influencé le cadre de mon stage, qui a abouti à la conception d’un pipeline de données pour la reprise des reportages photographiques de la Présidence de la République en vue d'un versement dans le SAE Vitam. Ce parcours permet également de comprendre les traitements appliqués aux fonds archivés avant l'ère Vitam, mais aussi de poser les bases de notre réflexion sur la manière dont les paquets d'archives doivent être construits pour répondre aux exigences de ce nouveau système d'archivage.

\subsection*{L'aube de l'archivage électronique aux Archives nationales : le programme Constance}

Les Archives nationales de France ont entrepris la collecte d’archives numériques dès 1982 grâce au programme Constance (CONServation et Traitement des Archives Nouvelles Constituées par l’Électronique), un projet pionnier qui a défini la politique, les processus, et les méthodes de traitement et de conservation des données numériques et de leurs métadonnées. Par extension, le service chargé de l’archivage électronique au Centre des Archives Contemporaines (CAC) du site de Fontainebleau a aussi été surnommé \emph{Constance}. Pendant plus de 30 ans, Constance a permis de collecter et de préserver les données issues d'enquêtes statistiques, avec des processus tels que la gestion des fichiers et de leurs métadonnées dans une base documentaire et leur archivage sur des bandes magnétiques \gls{lto}, en planifiant régulièrement des migrations de support. Des conversions de fichiers ont également été opérées dans une optique de conservation des documents bureautiques, notamment la conversion de fichiers Word au format PDF alors réputé plus pérenne \footcite{levasseurRetourExperienceStrategie2022}. 

Pour parer aux risques d'obsolescence technologique, l'équipe de Constance a adopté des solutions visant à assurer la pérennité des fichiers numériques, \enquote{quelle que soit leur forme technique}\footcite[p.62]{conchonArchivageFichiersInformatiques1988}, bien que les mutations technologiques incessantes mettent à mal cette noble tentative. Les données sont stockées sur des bandes LTO, un support plus durable et offrant de plus grandes capacités que les disques optiques. Néanmoins, les CD et DVD, apparus après les bandes LTO, ont été recommandés pendant les années 2010 pour la conservation à des fins d'archivage. Cependant, les migrations régulières vers de nouveaux supports demeurent inévitables : il n'existe pas à ce jour de support qui ne se dégrade pas au fil du temps. A titre d'exemple, les bandes LTO sont réputées fiables sur une période de 15 à 30 ans, bien que des migrations à des échéances plus courtes soient nécessaires afin de minimiser les risques de pertes de données\footcite{verlhiacQuEstceQue2023}. La durée de vie des CD et DVD dépasse rarement 10 ans, tandis que celle des disques durs externes de type HDD est estimée entre 5 et 7 ans. Sur ces bandes LTO, les fichiers sont conservés \enquote{à plat}, c'est-à-dire sans structure ou classement issus de logiciels métier ou d'une arborescence antérieure de dossiers et de sous-dossiers. 

Le programme Constance a également mis en place un système de nommage des fichiers, qui inclut plusieurs métadonnées, telles que le numéro de notice du producteur, le numéro d'entrée, le numéro d'article, le nom du fichier d'origine, et son extension, afin de faciliter leur identification et leur traçabilité dans le temps. Voici, par exemple, le nommage d'un fichier issu des reportages photographiques traités sous la mandature de Jacques Chirac : 

\begin{displayquote}
	\begin{center}
		009918\_20100562\_3546\_35460001.JPG
	\end{center}


Il s'agit ici d'un fichier au format JPEG, la première photographie du reportage 3546 (nom du fichier d'origine), dans le dossier correspondant au reportage 3546 (numéro d'article), de l'entrée 20100562, versé par le service photographique de la Présidence de la République, dont le numéro de notice producteur est 009918 (FRAN\_NP\_009918).
\end{displayquote}

Le programme Constance a évolué, passant de la conservation de données structurées issues d'applications informatiques, principalement des statistiques, à la prise en charge de nouveaux types de documents numériques, tels que les fichiers bureautiques, les messageries, les images, les vidéos et les documents sonores. Ce changement a été particulièrement marqué après 2010, lorsque les archives nativement numériques provenant des administrations centrales ont commencé à dominer les versements, rendant obsolètes certaines pratiques du programme initial\footcite{sinblimabarruArchivageNumeriqueAux2015}. En réponse à ces évolutions, le service Constance s'est réorganisé en 2012 au sein du Département de l'archivage électronique et des archives audiovisuelles (DAEAA) des Archives nationales, pour mieux gérer la diversité croissante des formats et des types de données.

Le projet ADAMANT (Administration des Archives et de leurs Métadonnées aux Archives Nationales, dans le Temps), lancé en 2015, a été conçu pour faire évoluer les pratiques d'archivage électronique aux Archives nationales, en réponse à l'inadéquation croissante du programme Constance face aux enjeux de l'archivage numérique des années 2010\footcite[p.220]{marcotteArchivesConduiteChangement2015}. En particulier, Constance ne permettait plus de gérer efficacement l'intégration des fonds physiques et numériques, une exigence devenue centrale avec l'augmentation des archives nativement numériques. ADAMANT s'inscrit ainsi comme un projet organisationnel et d'accompagnement au changement, visant à ne plus séparer la responsabilité des fonds en fonction de leur support. Ce projet a conduit à l'ouverture du SAE des Archives nationales en 2018 et à la création du Département de l'administration des données (DAD), qui a pris la suite du DAEAA.
\\

La méthode Constance est maintenue jusqu’à l’ouverture du système d'archivage électronique des Archives nationales en novembre 2018. Cependant, l'application des traitements étant un processus très chronophage, avec le renommage des fichiers et leur mise à plat, l'ensemble des archives versées avant 2018 n'a pas pu être traité intégralement. C'est notamment le cas des reportages photographiques de la Présidence de la République et des Services du Premier ministre, dont seulement une partie a été traitée. Après la décision de passer au système d'archivage électronique Vitam, il n'était plus nécessaire -- voire contre-productif -- de continuer l'application de la méthode sur les documents versés.

\subsection*{Le SAE Vitam}

Le programme Vitam (Valeurs immatérielles transmises aux archives pour mémoire), lancé officiellement le 9 mars 2015, est un projet interministériel d’archivage électronique conçu pour répondre aux défis contemporains de la gestion massive de documents numériques. Développé par trois ministères (Affaires étrangères, Culture, Armées) sous la supervision du Comité interministériel aux Archives de France et de la Direction interministérielle du Numérique, Vitam vise à proposer une solution logicielle libre, capable de traiter de larges volumes de documents nativement numériques de tout type (bureautiques, audiovisuels, bases de données). La solution logicielle doit garantir l'intégrité et la pérennité (respect de la valeur probante) des documents numériques, leur sécurité (duplication des serveurs, cybersécurité, souveraineté des espaces de stockage) et leur facilité d’accès pour un usage fréquent. 

Déployé progressivement entre 2015 et 2023 au sein des ministères porteurs à travers des plateformes adaptées comme \emph{Saphir} pour le ministère des Affaires étrangères, \emph{Archipel} pour le ministère des Armées et \emph{Adamant} pour les Archives nationales, Vitam se concentrait initialement sur des applications de backoffice. La conception des interfaces utilisateur était laissée à la charge de chaque institution, selon ses besoins spécifiques. Le programme Vitam s’inscrit dans une démarche collaborative, avec une communauté d’utilisateurs actifs qui a contribué à son évolution. Cette approche a mené à l’élaboration de Vitam UI, dont le développement a débuté en 2019, pour répondre aux besoins de nouveaux utilisateurs n'ayant pas les moyens de créer leur propre interface. L’ensemble des exigences fonctionnelles découlant du cadre normatif évoqué dans le chapitre précédent a orienté le fonctionnement et l’architecture de Vitam : la norme NF Z 45-013 pour le système d’archivage électronique (SAE), la norme OAIS pour les interactions et la traçabilité, et le format SEDA pour la modélisation de l'ensemble des transactions définies par la norme OAIS.

Conformément au modèle conceptuel OAIS, la solution logicielle Vitam prend en entrée des paquets d’informations (Submission Information Packages, ou SIP). Or, cette dernière ne permet pas de générer les SIP à partir des fichiers et de leurs métadonnées. Cette fonction est déléguée à un outil tiers, appelé ReSIP, intégré dans un second temps à l'architecture Vitam et téléchargeable sur le site officiel de la solution logicielle. Une fois le SIP constitué à l’aide de l’outil ReSIP, il se présente sous la forme d’un conteneur (.zip ou .tar) comprenant un répertoire contenant l’ensemble des objets numériques mis à plat, ainsi qu’un bordereau, communément appelé le \emph{manifest}. Ce document contient l’ensemble des métadonnées descriptives
et informations de pérennisation décrites dans le chapitre précédent : il permet notamment de reconstituer l’arborescence des objets numériques après leur ingestion dans Vitam et d’attribuer une empreinte unique à chaque
objet.
\\

Comme évoqué dans l'introduction de ce mémoire, l'application conçue au cours de mon stage au Département de l'administration des données (DAD) s'inspire des fonctionnalités de ReSIP (mise à plat des fichiers, écriture du manifest) en s'adaptant aux besoins spécifiques de la reprise des reportages photographiques par l'ajout d'une étape d'extraction des métadonnées internes des fichiers afin d'enrichir le signalement des versements dans le SIA des Archives nationales. Il est important de préciser cependant qu'à ce jour, le SIA des Archives nationales ne permet pas la recherche par mots-clés : s'ils apparaissent bien dans l'interface, ils ne peuvent être recherchés par le moteur de recherche ou à l'aide de filtres à facettes. Seuls quelques champs sont interrogeables, notamment le titre de l'unité d'archive et sa cote. Cette limitation ne vient pas du SAE Vitam mais bien de l'interface des Archives nationales qui ne rend pas interrogeable l'ensemble des métadonnées descriptives du paquet. Cette spécificité découle d'un choix effectué au moment de la conception de cette partie du SIA,, choix qui pourrait être révisé pour améliorer l'interrogabilité des fonds. Cette possibilité nous a conforté dans notre décision d'exploiter les métadonnées internes des photographies dans l'hypothèse où une mise à jour du SIA permettra à terme d'étendre les champs interrogés par le moteur de recherche.

\section{Le chantier de reprise des données des reportages photographiques de la Présidence et des services du Premier ministre}

Le travail produit dans le cadre de mon stage s'inscrit dans le contexte plus global du chantier de reprise des données conservées sur disques-durs et sur bandes LTO vers le nouveau système d'archivage électronique Vitam des Archives nationales. Après une présentation de la reprise des données, nous nous intéresserons aux spécificités des fonds de reportages photographiques qui ont nécessité l'élaboration de plusieurs méthodes de traitement. La méthode de reprise semi-automatique des reportages des services du Premier Ministre sera ensuite présentée : elle correspond à un premier état des réflexions qui ont mené, dans un second temps, à la conception du pipeline de données qui automatise le traitement d'une partie des reportages photographiques de la Présidence de la République.

\subsection*{Présentation du projet}

Le passage de la méthode Constance au SAE Vitam entraîne non seulement un changement de méthode, mais aussi une évolution du support de stockage des archives numériques. Les archives numériques versées avant la mise en place du SAE sont stockées sur des bandes LTO ou sur des disques durs pour les dernières entrées non traitées. Cet état de fait n'est satisfaisant ni sur le plan de la communication ni sur celui de la sécurisation, les archives n'étant pas accessibles aux divers utilisateurs et les supports matériels se dégradant au fil du temps. Le chantier de reprise des données a donc pour objectif de transférer l'ensemble des documents versés avant la mise en place du SAE Vitam vers cette nouvelle plateforme d'archivage électronique. Les reportages photographiques concernant les mandatures de Jacques Chirac, Nicolas Sarkozy et François Hollande ayant été versés avant cette migration, ils font partie de cet ensemble à reprendre.

Le chantier de reprise des données mis en place en 2019 par Martine Sin Blima-Barru, responsable du DAD, a été confié à Émeline Levasseur\footnote{La description du chantier de reprise des données développée dans cette partie s'appuie sur des documents de communication internes produits au sein du service : \enquote{Retour AMOA reprise des données structurées}, présentation PowerPoint du 7 octobre 2019, et \enquote{La reprise des données et des métadonnées formant le patrimoine numérique des Archives nationales}, présentation PowerPoint du 18 mars 2021.}. La première étape du processus de reprise a consisté en l'analyse des données transférées sur serveurs après avoir été extraites des bandes LTO ou des disques durs. Pour rendre accessibles leurs métadonnées, la cartographie des fonds à reprendre devait permettre de lister les sources d'informations et de les associer aux archives correspondantes : exports CSV des bases de données Cindoc, instruments de recherche publiés ou non, etc. L'association entre ces métadonnées externes et les fichiers nécessitait l'identification de valeurs pivots telles que le numéro de versement ou le numéro d'article. Les métadonnées disponibles ont été divisées en plusieurs catégories en fonction de leur utilité : métadonnées descriptives, métadonnées techniques, métadonnées de gestion. Cette étape de cartographie a également permis de distinguer les métadonnées disponibles de celles qui restaient à déterminer (nombre de fichiers, poids, formats des données).

La cartographie des données et des métadonnées a permis leur catégorisation en différentes typologies homogènes : bureautique, photo-audiovisuel, données structurées, messageries, archives orales. L'intérêt de cette catégorisation étant d'envisager des méthodologies de traitement similaires pour chaque ensemble homogène. A l'identification de ces grands ensemble s'ajoutent des préoccupations archivistiques qui ont permis de déterminer des priorités quant à l'ordre des reprises : le besoin de rendre accessibles les fonds communicables, la nécessité d'avoir repris les versements auxquels s'ajouteront de nouvelles entrées, la prise en compte de l'espace occupé sur les serveurs.

A l'intérieur de chacun des ces grands ensembles, il peut ensuite être nécessaire d'établir une catégorisation plus fine afin de déterminer si une seule méthode de reprise peut être appliquée à l'ensemble des données, ou s'il sera nécessaire d'adapter la méthode à des spécificités propres à chaque versement. Au cours de mon stage de première année de master en 2023, j'ai effectué une première analyse des fonds de reportages photographiques de la Présidence de la République et des services du Premier ministre dans cette perspective. Cette première analyse, confortée lors de mon stage de master 2 en 2024, a révélé l'hétérogénéité de ces reportages. Il convenait alors de développer des méthodes spécifiques à chaque service producteur, mais également au sein même des versements.


\subsection*{Hétérogénéité des fonds}

Les évolutions des pratiques d’archivage électronique aux Archives nationales ont engendré une hétérogénéité notable dans le classement des fonds de reportages photographiques. Les reportages traités selon la méthode Constance, caractérisée par le renommage systématique des fichiers et leur mise à plat, contrastent avec ceux qui ont été partiellement ou non traités et qui présentent une organisation plus variée et moins uniforme. Dans ce contexte, il apparaît pertinent de différencier la reprise des reportages des services du Premier ministre de celle des reportages de la Présidence de la République. Les reportages du Premier ministre sont relativement homogènes, ayant été largement traités suivant la méthode Constance. De plus, leur volumétrie reste plus modeste et donc plus facilement appréhendable par rapport aux fonds des reportages de la Présidence. Cette homogénéité justifie l'adoption d'une méthode spécifique de fabrication des SIP et d’ingestion pour l’ensemble de ces reportages. Un enjeu central de ce processus réside dans la gestion de l’arborescence des dossiers : faut-il récupérer une arborescence existante ou reconstituer une arborescence perdue mais conservée intellectuellement dans le nommage des fichiers ? La réponse à cette question déterminera nécessairement des approches méthodologiques distinctes.

Pour les reportages de la Présidence, divers scénarios ont été identifiés, chacun présentant des caractéristiques distinctes :

\begin{enumerate}
\item Reportages de la mandature de Jacques Chirac non traités 1 : Un dossier par reportage, subdivisé en plusieurs sous-dossiers pour chaque séquence. Les fichiers, souvent non renommés, peuvent être organisés de manière complexe avec plusieurs niveaux de sous-dossiers, incluant à la fois des séquences et des sélections de photographies.
\item Reportages de la mandature de Jacques Chirac non traités 2 : Un dossier par reportage avec des fichiers non renommés à plat, accompagné d’un ou plusieurs dossiers de photographies sélectionnées.
\item Reportages de la mandature de Jacques Chirac traités : Un dossier par reportage avec les fichiers renommés à plat. Une arborescence disparue peut être reconstituée à partir des noms de fichiers.
\item Reportages de la mandature de Nicolas Sarkozy traités : Un dossier par mois avec les fichiers renommés à plat. Une arborescence disparue peut être reconstituée au sein d’un même reportage à partir des noms de fichiers.
\item Reportages de la mandature de Nicolas Sarkozy non traités : Un dossier par mois puis un dossier par reportage, avec plusieurs niveaux de sous-dossiers pour diviser les reportages en séquences. Les fichiers ne sont pas renommés.
\item Reportages de la mandature de François Hollande : Un dossier par année puis un dossier par reportage, avec plusieurs niveaux de sous-dossiers pour organiser les séquences des reportages. Les fichiers ne sont pas renommés.
\end{enumerate}

Lors de mon stage de Master 2, il a été décidé de développer une méthode spécifique, suivie par la création d’une application (ou pipeline de données) sur le fonds le plus homogène, à savoir celui de François Hollande. Cette application vise à restituer une arborescence déjà existante plutôt qu’à la recréer à partir du nommage des fichiers, répondant ainsi aux besoins particuliers de ce fonds. Cette application pourra être ensuite utilisée pour la reprise des autres reportages non traités, ou servir de base à une future méthode de traitement.

D’autres différences entre les fonds de la Présidence et du Premier ministre ont également mené à des solutions d’archivage distinctes, notamment en ce qui concerne la richesse et la qualité des métadonnées internes. Le fonds du Premier ministre, avec une indexation et une description moins détaillées, justifie une approche différente. Il a été jugé moins pertinent de conserver une indexation au niveau de chaque photographie, surtout vu le faible niveau de granularité des descriptions, qui sont souvent associées à des groupes de photographies plutôt qu’à des photographies individuelles. Cette réflexion a conduit à la décision de remonter le contenu des mots-clés au niveau des reportages, un travail de normalisation rendu plus envisageable par la moindre richesse des mots-clés et la plus petite taille de ce fonds par rapport à celui de la Présidence.
\\

Dans la partie suivante, je présenterai en détail le processus de reprise des reportages des services du Premier ministre. Ce processus, bien que semi-automatique, a été crucial pour alimenter les réflexions sur l’automatisation des traitements des reportages présidentiels. 

\subsection*{Une solution semi-automatique pour les reportages photographiques des services du Premier ministre}

Les reportages photographiques du service du Premier ministre sont regroupés dans des paquets d’archives, ou SIP, générés par le logiciel ReSip. Ce processus s’appuie sur une méthode d’importation de données reposant sur un fichier CSV de métadonnées. Chaque paquet correspond à une période de l’année, suivant une logique volumétrique et regroupe l’ensemble des données à verser ainsi que les métadonnées associées. Avant la constitution des paquets, une réorganisation des fichiers a été nécessaire. Le nommage des fichiers ayant suivi la méthode Constance, cette standardisation du nommage a permis, grâce à des commandes Powershell, de reconstituer une arborescence de dossiers correspondant aux différents reportages. Cette étape a facilité l’organisation des fichiers selon le classement adopté dans l’instrument de recherche.

La méthode de constitution des paquets est uniforme pour tous les reportages photographiques des services du Premier ministre, incluant le choix des métadonnées conservées ou créées lors de la reprise des données. Ce processus semi-automatisé repose sur la création de tableurs, sous forme de fichiers CSV de métadonnées, qui sont ensuite importés dans ReSip. Ce modèle de tableur peut être appliqué à l’ensemble des paquets à créer, garantissant ainsi une méthode reproductible et efficace. Pour constituer un paquet, il est nécessaire de préparer un CSV de métadonnées inventoriant les fichiers de données et leurs métadonnées descriptives. Ces CSV sont réalisés manuellement à partir de deux sources se présentant elles-mêmes sous la forme de tableurs : l’instrument de recherche et les métadonnées internes des photographies, extraites avec l’outil ExifTool. Seules certaines informations issues de ces sources sont migrées vers le système d’archivage électronique : les dates et intitulés des reportages ainsi que les anciennes cotes, provenant de l’instrument de recherche, et les dates et heures de prise de vue, noms des photographes, informations de localisation et mots-clés issus des métadonnées internes des photographies.

Les informations de localisation et les mots-clés, souvent associés à un ensemble de photographies ou à l’ensemble d’un reportage, sont reportés au niveau du reportage pour assurer une indexation plus pertinente et fiable. Les noms des photographes, lorsqu’ils sont disponibles, sont documentés au niveau de chaque photographie pour une meilleure traçabilité, notamment dans les cas où plusieurs photographes ont couvert le même reportage. Toutes ces informations sont ensuite nettoyées, soit directement dans Excel, soit dans \gls{openrefine}. Une fois les CSV de métadonnées constitués, ils sont importés dans ReSip, qui génère les SIP destinés à être transférés dans le SAE des Archives nationales. 

Pour faciliter l’application de cette méthode, j’ai rédigé un guide détaillé, disponible en annexe, qui présente pas à pas chaque étape du processus décrit ci-dessus\footnote{Voir le pas à pas en annexe \ref{sec:annexe2}.}.
\\

Dans ce chapitre nous avons présenté le cadre normatif et institutionnel de l'archivage électronique aux Archives nationales. Cette présentation a permis d'expliciter les règles auxquelles devra se conformer l'application de reprise des reportages photographiques de la Présidence, mais aussi les traitements antérieurs effectués sur les archives à reprendre. En effet, le chantier de reprise des archives numériques doit prendre en compte les traitements induits par la méthode Constance et les exigences de la nouvelle solution d'archivage Vitam. Si les reportages photographiques des services du Premier ministre, intégralement traités, ont pu être repris en suivant une seule et même méthode, ce n'est pas le cas des reportages de la Présidence de la République qui présentent plusieurs états de traitement. Nous avons présenté la méthode semi-automatique de reprise des reportages des services du Premier ministre, qui convenait à un fonds homogène et d'un volume relativement réduit. En revanche, l'hétérogénéité du classement des reportages de la Présidence, ainsi que sa volumétrie bien supérieure, ne sont pas compatibles avec une solution de reprise semi-automatique. Face à la masse des données, il devient nécessaire d'envisager des solution d'automatisation qui intègrent les apports qualitatifs d'un traitement manuel, notamment l'indexation au niveau des fichiers. 