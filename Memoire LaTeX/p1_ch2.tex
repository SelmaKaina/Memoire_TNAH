\chapter{Les caractéristiques des photographies numériques : analyse des spécificités techniques pour une gestion efficace des archives}

Pour comprendre les enjeux liés à la reprise des reportages photographiques et les défis rencontrés lors de la conception du pipeline de données, il est indispensable de se familiariser avec les caractéristiques techniques des fichiers numériques, et plus particulièrement des photographies. Ce chapitre vise à approfondir cette compréhension en présentant les éléments techniques qui influencent la gestion et l'intégrité des archives photographiques nativement numériques. Nous commencerons par une présentation des caractéristiques d'un fichier numérique, en mettant l'accent sur les spécificités des photographies numériques. Cette présentation fournira une base solide pour comprendre les enjeux techniques associés à la gestion des images. Nous examinerons ensuite les différents formats de fichiers photographiques identifiés dans les reportages de la Présidence de la République. Nous décrirons leurs caractéristiques en termes de qualité, de compression, et de flexibilité. Le chapitre se poursuivra avec une description du concept de métadonnées, un aspect crucial des photographies numériques. Nous explorerons les métadonnées contenues dans les photos de la Présidence et leur impact sur les utilisations possibles des fichiers. Enfin, nous aborderons le concept d'empreinte numérique et ses usages dans le contexte de l'archivage numérique.

\section{Comprendre la composition des archives iconographiques numériques}
\subsection*{Le binaire : le codage des fichiers numériques}
Les fichiers numériques apparaissent différemment à nos yeux et au processeur de notre ordinateur. Alors que nous voyons du texte, une image ou un tableur, l'ordinateur interprète ces informations en langage binaire, une suite de 0 et de 1. Ce langage binaire correspond au fonctionnement des processeurs, composés de milliards de transistors, sortes d'interrupteurs qui ne reconnaissent que deux états : le courant passe (1) ou ne passe pas (0). L'unité de base est le \emph{bit}, et 8 bits forment un \emph{octet}, permettant de représenter 256 valeurs différentes. Chaque fichier numérique est donc une séquence de 0 et de 1, structurée pour être interprétée par un logiciel, afin de la rendre intelligible pour nous.

\subsection*{L'images matricielle : une mosaïque de 0 et de 1}

Les images numériques se classent en deux grandes catégories : les images vectorielles et les images matricielles. Les photographies numériques relèvent de cette seconde catégorie. Une image matricielle se compose d'une mosaïque de pixels, similaires à de minuscules carreaux colorés, qui, lorsqu'ils sont vus de loin, créent une image cohérente. Cependant, en s'approchant, chaque pixel devient distinct, révélant l'illusion d'une continuité visuelle. La qualité d'une image numérique dépend de deux notions clés : la résolution et la définition. La résolution se réfère au nombre de pixels par unité de longueur, souvent exprimée en pixels par pouce (PPP ou DPI), tandis que la définition concerne les dimensions de l'image en pixels. Une résolution élevée signifie plus de détails, donc une meilleure qualité visuelle\footcite{clirDidacticielImagerieNumerique2000}.

Chaque pixel d'une image numérique est codé en langage binaire. La profondeur des couleurs, ou codage des couleurs, détermine combien de bits sont utilisés pour représenter la couleur d'un pixel. Les images en niveaux de gris peuvent avoir une profondeur de 2 à 8 bits, alors que les images en couleur, généralement codées en RVB (rouge, vert, bleu), peuvent atteindre une profondeur de 24 à 48 bits, permettant ainsi la création de plus de 16 millions de couleurs. La taille d'une image numérique dépend directement de ses dimensions, de sa résolution, et de sa profondeur de bits. Plus ces paramètres sont élevés, plus le fichier est volumineux, ce qui a un impact direct sur son utilisation. Par exemple, une résolution de 300 DPI est recommandée pour une impression de haute qualité, tandis que 72 DPI suffisent pour l'affichage sur le web. Certains formats de fichiers sont conçus pour manipuler des images de très haute qualité, mais ils sont également plus gourmands en espace de stockage.

\section{Les formats de photographies numériques}

Nous savons désormais comment se compose une image numérique. Mais comment notre ordinateur sait-il sous quelle forme le fichier doit être restitué ? Quel logiciel sera en mesure de l'ouvrir, parmi les multiples logiciels installés ? Comment détermine-t-il qu'il s'agit d'une image, d'un document textuel ou encore d'une vidéo ? La réponse réside dans le format de fichier. Un format de fichier est une spécification qui détermine la structure des données à l'intérieur du fichier, notamment la manière dont les informations sont encodées, organisées et stockées. Chaque format de fichier possède une structure unique, reconnue par les logiciels et les systèmes d'exploitation.

\subsection*{Les méthodes d'identification de format}
Il existe trois méthodes d'identification de format : 
\begin{enumerate}
    \item\emph{L'extension du fichier.} L'extension est le suffixe à la fin d'un nom de fichier (.png, .csv, .pdf), indiquant à l'ordinateur le logiciel à utiliser pour l'ouvrir. Modifiable manuellement par l'utilisateur, elle n'affecte en rien la composition du fichier. De ce fait, cette méthode d'identification reste peu fiable.
    
    \item\emph{Le type MIME (Multipurpose Internet Mail Extensions).}\footcite{programmevitamIdentificationFormatsFichiers2020} Créé en 1991 par les laboratoires Bell Corporation, le type MIME permet à l'origine d'insérer des documents dans des messages électroniques, puis s'est étendu aux protocoles de transfert sur le web. Il s'agit d'une étiquette interne qui identifie le format du fichier en ligne, indiquant aux logiciels comment afficher les données. Composé d'un type et d'un sous-type séparés par un slash (image/jpeg, video/mp4), le type MIME est plus fiable que l'extension, mais présente des inconvénients, comme le partage d'un même type par plusieurs versions d'un format.
    
    \item\emph{La signature.} Il s’agit d’une ou plusieurs séquences de bits, que l'on retrouvera dans l'ensemble des fichiers encodés dans ce format. Elle est généralement positionnée à un endroit spécifique du fichier, souvent au début ou à la fin. Cette méthode d'identification est la plus fiable des trois, mais également la plus complexe à mettre en \oe{}uvre.
\end{enumerate}


\subsection*{Les formats identifiés lors de la reprise des données des reportages de la mandature de François Hollande}
A l'aide du logiciel d'identification de format DROID (Digital Record Object Identification), nous avons procédé à une analyse des formats de fichiers présents dans les versements de reportages photographiques de la Présidence de la République pour la mandature de François Hollande. Il s'agissait d'identifier non seulement les types de formats d'images numériques présents dans ce fonds, mais également les autres types de fichiers et les fichiers endommagés. En effet, les logiciels d'identification de format tels que DROID parcourent la séquence de bits qui compose les fichiers afin d'en extraire les informations permettant de déterminer le format du fichier en les comparant aux informations dun référentiel de format \gls{pronom} : si le format ne peut être identifié, cela peut signifier que la séquence est endommagée, par un ajout ou une absence d'informations par exemple, ou que le format n'existe pas dans la base de données PRONOM\footcite{pronom2024}. Nous nous concentrerons ici uniquement sur les formats d'images identifiés par DROID et sur leurs caractéristiques. Nous reviendrons sur les autres formats identifiés dans la deuxième partie de ce mémoire, lorsque nous aborderons le choix des fichiers exclus de la reprise des données.

\begin{table}[h]
\begin{tabular}{|p{5cm}|p{3cm}|p{2cm}|p{4cm}|}
    \hline
    \rowcolor{pastelpink-dark}
    \textbf{Nom du format} & \textbf{Identifiant PRONOM} & \textbf{Type MIME} & \textbf{Nombre de fichiers} \\
    \hline
    \rowcolor{pastelpink}
    Exchangeable Image File Format (Compressed) & fmt/645 & image/jpeg & 446 083 \\
    \hline
    \rowcolor{pastelpink}
     JPEG File Interchange Format & fmt/44 & image/jpeg & 2716 \\
     \hline
     \rowcolor{pastelpink}
     Canon RAW & fmt/592 & image/x-canon-cr2 & 230 \\
     \hline
     \rowcolor{pastelpink}
     Raw JPEG Stream & fmt/41 & image/jpeg & 67 \\
     \hline
 \end{tabular}
 \caption{Les formats d'images identifiés par le logiciel DROID dans les reportages photographiques de la mandature de François Hollande, entre 2012 et 2017}
 \end{table}

\subsection*{Les formats d'images numériques : caractéristiques et contextes d'utilisation}

Parmi les quatre formats d'images identifiés, trois sont liés au format JPEG et un est au format RAW. Le JPEG, développé par le Joint Photographic Experts Group de l'ISO, est le format d'image le plus courant. Il offre un bon équilibre entre qualité et taille de fichier, grâce à une compression avec perte modulée d'information.

\subsubsection*{La compression d'images numériques}

Qu'est-ce que la compression d'image ? Comme mentionné précédemment, la taille des fichiers image dépend de leur résolution et de la profondeur des couleurs. La compression vise à réduire cette taille pour en faciliter le stockage et la manipulation. Elle transforme la séquence de bits d'une image en une formule mathématique plus concise, à l'aide d'algorithmes spécifiques. Il existe deux types de compression : non destructive et destructive. La compression non destructive conserve toutes les informations, garantissant une image identique à l'original après décompression. En revanche, la compression destructive, utilisée notamment par le format JPEG, élimine certaines informations jugées moins importantes, en fonction de la perception visuelle humaine. Bien que cette méthode puisse réduire la qualité, les pertes sont souvent subtiles et difficiles à percevoir\footcite{clirDidacticielImagerieNumerique2000}. 

Le choix du format d'image dépend souvent de son usage spécifique. Par exemple, pour une image destinée à être affichée sur Internet, une basse définition peut suffire, l'objectif étant de garantir une restitution efficace du contenu informationnel sans ralentir le chargement de la page. En revanche, pour alimenter une base de données iconographique destinée à l'archivage d'images en haute définition, il est préférable de choisir un format sans compression ou avec compression sans perte. La cellule photographique de la Présidence de la République semble avoir privilégié le format JPEG pour ses photographies, avec une profondeur de 24 bits et une résolution de 72 dpi. Bien que ces images soient de bonne qualité et que la compression soit imperceptible à l'œil nu, leur résolution reste inférieure aux standards requis pour des impressions en grand format et haute qualité (300 dpi). Cependant, cette résolution correspond aux usages habituels des fichiers produits par la cellule, tels que la création de petits albums photographiques, l'illustration de supports de communication en ligne, et la documentation des activités présidentielles.

\subsubsection*{Présentation des formats identifiés : le RAW et le JPEG}

Les formats RAW se distinguent nettement des autres formats d'images, étant spécifiquement conçus pour la photographie numérique. Ces formats, souvent propriétaires, sont développés par les fabricants d'appareils photo numériques et peuvent être compressés sans perte, avec perte, ou même non compressés. Ils permettent d'encoder et d'afficher les clichés directement dans l'appareil photo. Qualifiés de \enquote{négatifs numériques} par analogie avec la photographie argentique, les fichiers RAW sont enregistrés directement depuis le capteur photosensible sans modification, conservant ainsi l'intégralité des informations colorimétriques et photométriques associées à la prise de vue\footcite[pp.79-80]{chirolletPenserPhotographieNumerique2015}. Par exemple, certains formats RAW, comme le CR2 de Canon, peuvent être non compressés, tandis que d'autres, comme le DNG d'Adobe ou certaines versions du CR3 de Canon, utilisent une compression sans perte. Lors de l'extraction de ces fichiers de l'appareil, ils sont souvent convertis en un format plus facilement manipulable, tel que le JPEG, un processus connu sous le nom de \enquote{développement} des fichiers RAW. Cependant, ces formats propriétaires posent des défis en matière d'interopérabilité et de lisibilité à long terme, car ils ne sont pas toujours compatibles avec tous les logiciels de visionnage d'images sur ordinateur. Pour remédier à cette limitation, Adobe a publié en 2004 les spécifications du format RAW universel DNG (Digital Negative) et a développé un logiciel permettant de convertir les fichiers RAW propriétaires en ce format universel\footnote{Voir la description du format sur le site Adobe (url : \url{https://helpx.adobe.com/fr/camera-raw/digital-negative.html}).}. En outre, les fichiers RAW sont souvent beaucoup plus volumineux que d'autres formats d'image, sans que la différence qualitative soit toujours perceptible à l'\oe{}il humain.

Le format JPEG semble avoir été privilégié par la cellule photographique de la Présidence de la République.  La présence de fichiers au format RAW dans les reportages est cependant difficile à justifier. Elle pourrait être intentionnelle : les photographes ont peut-être extrait délibérément des versions RAW de leurs images afin de les retoucher avec des logiciels spécialisés, le format RAW étant souvent privilégié pour le traitement d'images. Si tel est le cas et que ces retouches ont été effectuées, nous ne disposons pas des versions retouchées. Une autre hypothèse est que les fichiers RAW aient été exportés automatiquement lors du transfert des photographies de l'appareil à la photothèque. Dans ce cas, il est difficile d'évaluer aujourd'hui si cet export résulte d'un bug ou d'un paramétrage spécifique de des appareils photographiques.

\subsubsection*{Du négatif au fichier RAW : définir la notion d'original pour la photographie nativement numérique}

Ici déjà nous observons une différence significative avec la photographie nativement numérique : si un fichier RAW est un négatif numérique, l’ensemble des JPEG correspondant doivent-ils être considérés comme des tirages ? L’analogie montre ici ses limites, car un fichier RAW n’est souvent pas lisible sans une conversion préalable nécessitant des logiciels spécifiques. De plus, ces fichiers ne sont pas toujours exportés par les photographes et sont souvent absents des versements. La logique est donc inverse : les fichiers conservés en priorité sont ceux au format JPEG, tandis que les RAW ne sont conservés qu’en l’absence d’un équivalent JPEG. La notion de « copie », enfin, est très différente lorsqu’il est question de fichiers numériques. En effet, les propriétés des images numériques leur confèrent une forme \enquote{d’ubiquité visuelle}\footcite[p.16]{chirolletPenserPhotographieNumerique2015}. Lorsqu’un fichier numérique est copié, la copie reprend à l’identique l’intégralité du train binaire du fichier, au point qu’il est impossible de distinguer l’original de la copie : les deux fichiers auront le même format, les mêmes métadonnées, et donc la même empreinte. Seul le nommage du fichier, qui n’impacte pas la structure formelle du fichier, pourrait permettre de les distinguer. Une même photographie peut donc exister à plusieurs endroits à la fois, sous la forme de doublons techniques indifférenciables. 
\\

Les spécificités des formats de fichiers numériques, qu'ils soient compressés ou non, ont un impact direct sur la manière dont les images sont traitées et conservées. En parallèle des caractéristiques techniques des formats, un autre aspect central de la gestion des images numériques est la gestion des métadonnées. Les métadonnées, qui sont des informations additionnelles intégrées au sein des fichiers d'image, jouent un rôle essentiel dans la documentation et l'archivage des contenus numériques. Elles peuvent inclure des détails sur les paramètres de prise de vue, les droits d'auteur, ou encore des informations sur le contexte et l'utilisation des images. Ainsi, après avoir exploré les formats et leurs implications sur la qualité et la taille des fichiers, il est important de se pencher sur la manière dont les métadonnées internes enrichissent ces fichiers et facilitent leur gestion et leur conservation à long terme.

\section{Les métadonnées et leur rôle fonctionnel : de la description à la gestion documentaire}
Le terme \emph{métadonnées} signifie littéralement \enquote{données relatives aux données}. Dans le contexte de l'archivage électronique, les métadonnées sont définies comme « les données décrivant le contexte, le contenu et la structure des documents ainsi que leur gestion dans le temps »\footcite[p.121]{rietschDematerialisationArchivageElectronique2006}. Différents types de métadonnées se distinguent en fonction de leur usage, de leur contenu et de la manière dont elles sont créées. En termes d'usage, on peut distinguer les métadonnées descriptives des métadonnées techniques, ou encore les métadonnées de gestion administrative des métadonnées de conservation\footcite{clirDidacticielMetadata}. Plusieurs métadonnées techniques internes des photographies numériques sont créées au moment de la prise de vue, dans l'appareil photographique : la définition, la profondeur de couleur, le temps d’exposition, l’ouverture du diaphragme, l’intensité du flash, ou encore la taille du fichier et sa date de création. En termes de contenu, les métadonnées structurelles indiquent comment les différentes parties d'un document ou d'un ensemble de documents sont liées entre elles et comment elles doivent être présentées ou naviguées, tandis que les métadonnées contextuelles permettent de comprendre les circonstances entourant la création d'un document, son utilisation, ou encore les modifications qu'il a subies au fil du temps\footcite[pp.121-122]{rietschDematerialisationArchivageElectronique2006}. Enfin, les métadonnées elles-mêmes peuvent être créées dans des contextes différents. Pour une photographie numérique, par exemple, certaines métadonnées sont ajoutées manuellement par un iconographe dans une photothèque.

Certaines métadonnées sont dites \enquote{internes} car elles sont intégrées au train binaire du fichier. D'autres métadonnées sont dites \enquote{externes} car elles sont issues d'un autre document. Dans le contexte archivistique, les informations renseignées dans un instrument de recherche peuvent être considérées comme des métadonnées externes décrivant les fichiers du fonds. 

\subsection*{Les schémas de métadonnées des images numériques : Exif, XMP, IPTC}

Les métadonnées des photographies numériques se déclinent en plusieurs schémas, chacun ayant des caractéristiques distinctes et parfois des recoupements.

Le format EXIF (Exchangeable Image File Format) est couramment utilisé pour stocker les métadonnées dans les fichiers image comme JPEG et TIFF. EXIF intègre des informations techniques essentielles telles que la résolution et les paramètres de prise de vue dans l’en-tête du fichier sous forme de paires nom-valeur. Cependant, certains formats d'image ne prennent pas en charge le format EXIF, ce qui limite son utilisation\footcite[][pp.9-16]{chirolletPenserPhotographieNumerique2015}.

Le schéma IPTC (International Press Telecommunications Council), développé pour les agences de presse, se concentre sur les aspects descriptifs comme les crédits, les légendes, et les mots-clés\footcite{PresentationConceptsMetadonnees}.

Le format XMP (Extensible Metadata Platform) d’Adobe est un standard extensible pour le traitement et l’échange de métadonnées. XMP est compatible avec divers logiciels et formats, intégrant les données d'autres sources tout en permettant l’ajout d'informations spécifiques\footcite{PresentationConceptsMetadonnees}.

Chaque métadonnée, comme la date de création, peut apparaître dans plusieurs schémas, chacun ayant ses propres méthodes de stockage et d'organisation.

\subsection*{Cartographie des métadonnées internes des photographies des reportages de la Présidence de la République}

Comme toutes les images numériques, les photographies prises par la cellule photographique de l'Élysée contiennent des métadonnées techniques et descriptives. Certaines métadonnées descriptives, telles que le nom du photographe, peuvent être configurées directement sur l'appareil et appliquées par défaut à tous les fichiers générés. Cependant, la majorité des métadonnées descriptives sont ajoutées après l'export des photos vers une photothèque ou un logiciel de gestion d'images. Les métadonnées peuvent alors être ajoutées en masse à un groupe de fichiers ou individuellement pour chaque fichier. Les métadonnées descriptives les plus couramment renseignées dans les photographies de la Présidence de la République sont : le nom du photographe, le copyright, le lieu de la prise de vue (ville et/ou pays), une description de l'image, ainsi que des mots-clés.

Le choix des métadonnées descriptives dépend de la méthodologie de l'équipe d'indexation et du logiciel utilisé, ce qui peut évoluer avec le temps. Certains logiciels peuvent privilégier un schéma de métadonnées spécifique ou synchroniser les informations à travers différents schémas. Par exemple, une légende peut être renseignée dans le champ XMP mais pas dans le champ IPTC. Une analyse des métadonnées est donc essentielle pour identifier précisément les informations intégrées. Des variations ont été notées au sein du fonds photographique de la Présidence de la République, reflétant des changements dans les pratiques ou les outils utilisés. 

\subsection*{Interopérabilité et compatibilité des encodages de métadonnées}

Lorsqu'un projet exige l'exploitation des métadonnées internes d'un fichier, des problèmes d'encodage des métadonnées peuvent survenir. L'encodage des métadonnées se réfère à la méthode de conversion des caractères des métadonnées textuelles ou chiffrées (mots-clés, date de création, résolution) en séquence binaire pour le stockage dans le fichier. Le système d'encodage détermine le nombre d'octets nécessaires pour représenter les caractères. Par exemple, le Latin-1 utilise un octet par caractère, limitant ainsi le nombre de caractères représentables, tandis que l'UTF-8, largement utilisé, emploie de 1 à plusieurs octets par caractère selon sa fréquence. L'UTF-16, utilisant généralement 2 octets par caractère, est souvent utilisé dans les environnements multilingues. Le choix de l'encodage est important car il affecte la compatibilité des métadonnées avec différents systèmes et la représentation des caractères spéciaux, notamment dans les métadonnées multilingues. L'UTF-8 est préféré en raison de son efficacité pour représenter divers caractères tout en optimisant l'espace mémoire. Les systèmes anciens peuvent utiliser des encodages comme le Latin-1, nécessitant parfois une migration pour garantir l'affichage correct des caractères spéciaux.

Dans l'analyse des métadonnées des photographies de la cellule photographique de l'Élysée, des variations d'encodage ont été notées. Les métadonnées des reportages sous Jacques Chirac étaient en Latin-1, tandis que celles sous Nicolas Sarkozy et François Hollande étaient en UTF-8. Bien que cette différence semble mineure, elle impacte considérablement le traitement archivistique, nécessitant des processus distincts pour exploiter correctement les métadonnées internes selon les encodages.


\section{L'empreinte de fichier : la clé pour vérifier l'intégrité numérique}

Les métadonnées internes sont intégrées dans le fichier binaire et constituent une partie essentielle de celui-ci. Lors de l'analyse des images numériques, le fichier peut être divisé en deux grandes parties : les métadonnées internes et l'encodage des pixels, qui forme l'image affichable. Même si deux images ont des suites de pixels identiques, leurs métadonnées internes peuvent varier, rendant chaque fichier unique. Ces différences sont souvent invisibles lors de l'ouverture du fichier, et les logiciels de modification des métadonnées ne conservent généralement pas de trace des modifications. Pour garantir l'intégrité d'un fichier et détecter d'éventuelles altérations, il est essentiel de calculer son empreinte numérique à différents moments du processus de traitement.

Les empreintes numériques, ou \emph{hash}, sont des codes uniques dérivés du contenu des fichiers. Un algorithme de hachage prend le contenu binaire d'un fichier et produit une chaîne de caractères alphanumériques fixe. Cette empreinte est spécifique au fichier : un changement, même minime, modifie l'empreinte de manière significative. L'algorithme divise le binaire du fichier en séquences, traite chaque séquence par des opérations mathématiques et cryptographiques, puis combine les résultats pour créer une empreinte unique. 

Chaque algorithme de hachage génère une empreinte distincte pour un même fichier selon la méthode employée. Ces algorithmes sont conçus pour garantir une empreinte constante pour un fichier identique, même après de nombreuses années. Cependant, il est impossible de reconstruire le contenu original du fichier à partir de l'empreinte. Les empreintes sont utilisées pour sécuriser les mots de passe, dans les signatures électroniques, et en archivistique pour vérifier l'intégrité des données et identifier les doublons techniques. Avec la croissance massive des archives numériques, l'identification des doublons est devenue un enjeu central pour les archivistes.
